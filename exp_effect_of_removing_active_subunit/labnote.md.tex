\PassOptionsToPackage{unicode=true}{hyperref} % options for packages loaded elsewhere
\PassOptionsToPackage{hyphens}{url}
%
\documentclass[]{article}
\usepackage{lmodern}
\usepackage{amssymb,amsmath}
\usepackage{ifxetex,ifluatex}
\usepackage{fixltx2e} % provides \textsubscript
\ifnum 0\ifxetex 1\fi\ifluatex 1\fi=0 % if pdftex
  \usepackage[T1]{fontenc}
  \usepackage[utf8]{inputenc}
  \usepackage{textcomp} % provides euro and other symbols
\else % if luatex or xelatex
  \usepackage{unicode-math}
  \defaultfontfeatures{Ligatures=TeX,Scale=MatchLowercase}
\fi
% use upquote if available, for straight quotes in verbatim environments
\IfFileExists{upquote.sty}{\usepackage{upquote}}{}
% use microtype if available
\IfFileExists{microtype.sty}{%
\usepackage[]{microtype}
\UseMicrotypeSet[protrusion]{basicmath} % disable protrusion for tt fonts
}{}
\IfFileExists{parskip.sty}{%
\usepackage{parskip}
}{% else
\setlength{\parindent}{0pt}
\setlength{\parskip}{6pt plus 2pt minus 1pt}
}
\usepackage{hyperref}
\hypersetup{
            pdftitle={Effect of removing fixed fraction of active subunits from a CaMKII/PP1 Switch},
            pdfauthor={Dilawar Singh},
            pdfborder={0 0 0},
            breaklinks=true}
\urlstyle{same}  % don't use monospace font for urls
\setlength{\emergencystretch}{3em}  % prevent overfull lines
\providecommand{\tightlist}{%
  \setlength{\itemsep}{0pt}\setlength{\parskip}{0pt}}
\setcounter{secnumdepth}{5}
% Redefines (sub)paragraphs to behave more like sections
\ifx\paragraph\undefined\else
\let\oldparagraph\paragraph
\renewcommand{\paragraph}[1]{\oldparagraph{#1}\mbox{}}
\fi
\ifx\subparagraph\undefined\else
\let\oldsubparagraph\subparagraph
\renewcommand{\subparagraph}[1]{\oldsubparagraph{#1}\mbox{}}
\fi

% set default figure placement to htbp
\makeatletter
\def\fps@figure{htbp}
\makeatother

\usepackage[]{ebgaramond}
\usepackage[]{chemfig}
\usepackage[]{pgfplots}
\usepackage[]{tikz}
\usepackage[]{siunitx}
\makeatletter
\@ifpackageloaded{subfig}{}{\usepackage{subfig}}
\@ifpackageloaded{caption}{}{\usepackage{caption}}
\captionsetup[subfloat]{margin=0.5em}
\AtBeginDocument{%
\renewcommand*\figurename{Figure}
\renewcommand*\tablename{Table}
}
\AtBeginDocument{%
\renewcommand*\listfigurename{List of Figures}
\renewcommand*\listtablename{List of Tables}
}
\@ifpackageloaded{float}{}{\usepackage{float}}
\floatstyle{ruled}
\@ifundefined{c@chapter}{\newfloat{codelisting}{h}{lop}}{\newfloat{codelisting}{h}{lop}[chapter]}
\floatname{codelisting}{Listing}
\newcommand*\listoflistings{\listof{codelisting}{List of Listings}}
\makeatother

\title{Effect of removing fixed fraction of active subunits from a CaMKII/PP1
Switch}
\author{Dilawar Singh}
\date{\today}

\begin{document}
\maketitle

Previously ( in experiment of
\texttt{../exp\_MANY\_VOXELS\_COUPLED\_WITH\_DIFFUSION/}), we saw that
CaMKII/PP1 switch's \textbf{ON} state is sensitive to removal of active
subunits \(x\) (due to diffusion of \(x\) out of voxel/compartment ). In
this labnote, we quantify the effect of removing a fraction of active
subunits \(x\) from the system at rate comparable to diffusion of
\(x\).\footnote{In a previous version of this experiment, we removed all
  \(x\) at different rate. See
  \href{https://labnotes.ncbs.res.in/bhalla/removing-active-subunit-x-camkiipp1-increases-state-residence-time-1}{this
  labnote}. Needless to say the effect of removing subunits was rather
  drastic. This labnote creates more realistic situation by removing a
  certain fraction of \(x\). The diffusion process which is
  bi-directional is unlikely to remove all \(x\) from one system without
  replacing many of them with other \(x\)s from neighbouring system. }

In this experiment, we quantify this effect.

\hypertarget{summary-of-this-labnote}{%
\section{Summary of this labnote}\label{summary-of-this-labnote}}

Contrary to previous experiment (see footnote), we found that removal of
active subunit does not have very strong impact on switch unless all of
active subunits have been removed.

\begin{tikzpicture}[scale=1]
    \begin{axis}[
    xlabel=Fraction of $x$ removed,ylabel=Time spent in OFF state
    , grid style={draw=gray!20}, grid = both
    ]
    \addplot+ [color=blue] gnuplot [ raw gnuplot ] {
        plot "./summary.dat" using 1:2 with p;
    };
    \end{axis}
\end{tikzpicture}

\hypertarget{experiment-design}{%
\section{Experiment design}\label{experiment-design}}

We setup additional reaction which converts \(x\) into a dummy molecule
\(xs\). Molecules \(xs\) does not play any role in system. The forward
rate is fixed to 1 and the backward rate is changed in each run.

\[\schemestart x \arrow{<=>[$k_f$][$k_b$]} xs \schemestop \label{reac:1} \]

Above reaction removes a certain fraction of \(x\) by converting it into
\(xs\). This fraction is given by \(\frac{k_f}{k_f+k_b}\). This
`approximates' the effect of removing active subunits from the
CaMKII-PP1 switch by diffusion.

For a diffusion coefficient of \(D\) in a 1-D compartment of length
\(l\), the rate constant \(k_f = \frac{D}{l^2}\). For a typical value of
\(D=\SI{1}{\micro \meter^2 \per \second}\) in a 1-D compartment of
length \SI{1}{\micro \meter}, \(k_f = 1\).

\hypertarget{model-script}{%
\subsection{Model script}\label{model-script}}

Since this model have non-trivial modifications, the script is
maintained in this directory itself. If the reference model
\emph{../camkii\_pp1\_scheme.py} is changed significantly, then model
file in this directory should be changed accordingly.

\hypertarget{results}{%
\section{Results}\label{results}}

Following the plot of some runs.

\begin{@empty}
\newcommand{\fn}[1]{CaMKII-6+PP1-26+L-125e-9+N-1+diff-0_removal_kf1kb#1.dat_processed.dat}
\foreach \kb in {1000,10,1,0.1,0.005}
{
    Rate of reaction (see reaction \ref{reac:1} ) $k_f = 1$ and $k_b = \kb$.
    Fraction of $x$ removed \pgfmathsetmacro\rat{1/(\kb+1)} \rat.

    \edef\fileN{\fn{\kb}}

    \begin{tikzpicture}[scale=1]
    \begin{axis}[ xlabel=Time (days),ylabel=Active CaMKII
            , grid style={draw=gray!20}, grid = both, 
            , width = 0.6\linewidth, height = 4cm, scale only axis
            , ymax = 8, legend style={fill=none,draw=none}
        ]
        \addplot+ [no marks,thick] gnuplot [ raw gnuplot ] {
            plot "\fileN" using (column("time")/86400):"CaMKII*" every 50 with lines
        };
        \addplot+ gnuplot [ raw gnuplot ] {
            plot "\fileN" using (column("time")/86400):"x" every 50 with lines
        };
        \legend{CaMKII*, x}
    \end{axis}
    \end{tikzpicture} %
    \begin{tikzpicture}[scale=1]
    \begin{axis}[ xlabel=Active CaMKII
            , grid style={draw=gray!20}, grid = both, 
            , width = 0.4\linewidth, height = 4cm, scale only axis
            , ymax = 1.0
        ]
        \addplot+[no marks,fill=blue!20,hist={density,bins=7}] gnuplot [ raw gnuplot ] {
            plot "\fileN" using "CaMKII*" every 50;
        };
    \end{axis}
    \end{tikzpicture}
} 


\end{@empty}

\end{document}
