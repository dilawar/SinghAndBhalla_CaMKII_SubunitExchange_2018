% This is summary of CaMKII-PP1 pathway in SBGN-PD language. The layout is
% computed automatically using graphdrawing library.
%
% ./model_sbgn_pd.tex file has a manual layout.

\RequirePackage{luatex85}
\documentclass[crop,tikz]{standalone}
\usepackage[sfdefault,book]{FiraSans}
%\usepackage[default]{opensans}
\usetikzlibrary{calc,shapes,arrows,arrows.meta,positioning}
\usetikzlibrary{graphs,graphdrawing}
\usegdlibrary{force,trees,layered}
\usepackage{pgfplots}
\usepackage{sbgn}

% Include SBGN macros.

\begin{document}


\begin{tikzpicture}[ scale=1, node distance=1 cm]
    \large
    \def\xdist{1.5cm}
    \def\ydist{1cm}
    
    % Compute the layout here and put duplicate nodes by shapes.
    % First we compute the layout using the graphdrawing library; later we place
    % the nodes on top of computed node. For examples, a node X is compute here
    % and later we do
    %   \node (x) [at= X] {X};
    % This overwrite X by using the position of precomputed node X. 
    \graph [ 
        %layered layout, node distance=5mm, sibling sep=2cm
        %    , minimum layers=5 , level distance=2mm
        , tree layout, sibling distance=25mm, level distance=12mm
        %, spring  electrical layout, electric charge=10, random seed=10
        %, electric force order=2
        , nodes={as=}
        , edges={draw=none}  % If we draw them, they will not respect the boundries.
        , process/.style={draw,rectangle,blue,thick}
    ] 
    {
        { ca, cam } -- caCaMBinding[process] -- actCAM;

        camkiiInactive -- slowact[process] -- camkiiPartial;
        camkiiPartial -- fastact[process] -- camkiiFull;

        % Deactivation
        camkiiFull -- dephospho1[process] -- camkiiPartial --
            dephospho2[process] -- camkiiInactive;

        actCAM -- slowact;
        actCAM -- fastact;

        pp1 -- { dephospho2, dephospho1 };

        { pka, i1 } -- procI1Phospho -- i1p -- cplxI1PP1;


        { pp1, i1p } -- procI1PP1Cplx -- cplxI1PP1;

        % I1 exchange

        i1cytosol -- procI1Exchange -- i1;


        %% Following is only valid in layered layout
        %// { [same layer] camkiiInactive, camkiiFull, camkiiPartial, slowact, fastact }
        %// { [same layer] dephospho1, dephospho2 }
        %// { [same layer] pp1, i1, i1p, procI1PP1Cplx }
    };

    \SimpleMol{ca}{at (ca)}{Ca}
    \MacroMol{cam}{at (cam)}{CaM}{}
    \MacroMol{actCAM}{at (actCAM)}{CaM*}{}

    % This area belongs to activation of CaMKII.
    \def\nodeShift{15mm}
    \MacroMol{camkiiPartial}{at (camkiiPartial)}{CaMKII}{2P}
    \MacroMol{camkiiInactive}{at (camkiiInactive)}{CaMKII}{}
    \MacroMol{camkiiFull}{at (camkiiFull)}{CaMKII}{6P}

    % % Ca+CaM binding.
    \draw (ca) -- (caCaMBinding);
    \draw (cam) -- (caCaMBinding);
    \draw[output] (caCaMBinding) -- (actCAM);

    \draw[required stimulus] (actCAM) -- (slowact);
    \draw[stimulus] (actCAM) -- (fastact);

    \draw[catalyst] (camkiiPartial.east) to[bend right] (fastact);

    \draw (camkiiInactive) -- (slowact) edge[output] (camkiiPartial);
    \draw (camkiiPartial) -- (fastact) edge[output] (camkiiFull);


    %% dephosphorylation.
    \MacroMol{pp1}{at (pp1)}{PP1}{}

    \Connect{pp1Camkii6}{required stimulus}{pp1}{dephospho2}
    \Connect{pp1Camkii1}{required stimulus}{pp1}{dephospho1}

    \draw (camkiiFull) -- (dephospho1) edge[output] (camkiiPartial);
    \draw (camkiiPartial) -- (dephospho2) edge[output] (camkiiInactive);


    % I1 and I1P.
    \Process{procI1PP1Cplx}{at (procI1PP1Cplx)}{ }
    \MacroMol{i1p}{at (i1p)}{I1}{P};

    \Process{procI1Phospho}{at (procI1Phospho)}{}
    \MacroMol{i1}{at (i1)}{I1}{};
    \MacroMol{pka}{at (pka)}{PKA}{}

    \draw (i1) -- (procI1Phospho);
    \Connect{i1phospo}{output}{procI1Phospho}{i1p}
    \Connect{wirePkaI1Phospho}{catalyst}{pka}{procI1Phospho}

    % Make complex of I1P and PP1.
    \MacroMol{pp1a}{at (cplxI1PP1)}{PP1}{}
    \MacroMol{i1pa}{[right=of pp1a]}{I1}{P}
    \ComplexTwo{cplxI1PP1}{pp1a}{i1pa}{}

    \Connect{wireI1}{}{i1p}{procI1PP1Cplx}
    \Connect{wirePP1}{}{pp1}{procI1PP1Cplx}
    \Connect{wireCplx}{output}{procI1PP1Cplx}{cplxI1PP1}

    %% %%%%%%%%%%%%%%%%%%%%%%%%%%%%%%%%%%%%%%%%%%%%%%%%%%%%%%%%%%%%%%%%%%%%%%%%%%%
    %% %% Exchange of I1
    \MacroMol{i1Cyt}{at (i1cytosol)}{I1}{}
    \Place{cytosol}{at (i1cytosol)}{(cytosol)}{}

    \Process{procI1Exchange}{at (procI1Exchange)}{ }

    \Connect{x1}{ }{i1Cyt}{procI1Exchange}
    \Connect{x2}{output}{procI1Exchange}{i1}



\end{tikzpicture}    

\end{document}
