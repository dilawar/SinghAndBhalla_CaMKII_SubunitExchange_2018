% This is summary of CaMKII-PP1 pathway in SBGN-PD language.  %
\RequirePackage{luatex85}
\documentclass[crop,class=../elife]{standalone}
\usepackage{tikz}
\usetikzlibrary{graphs,graphdrawing}
\usegdlibrary{layered}
\usepackage{pgfplots}
\usepackage{sbgn}

\renewcommand\familydefault{\sfdefault}
\usetikzlibrary{calc,positioning}

\begin{document}
\begin{tikzpicture}[ scale=1, node distance=1 cm]
    \large
    \def\xdist{1.5cm}
    \def\ydist{1cm}

    \coordinate (origin) at (0,0);

    \SimpleMol{ca}{[below of=origin]}{Ca}
    \MacroMol{cam}{[right=of ca]}{CaM}{}
    \Process{caCaMBinding}{[below=of $(ca)!0.5!(cam)$]}{};

    \MacroMol{actCAM}{[below=of caCaMBinding]}{CaM*}{}

    % This area belongs to activation of CaMKII.
    \def\nodeShift{15mm}
    \MacroMol{camkiiPartial}{[below=of actCAM]}{CaMKII}{2P}
    \MacroMol{camkiiInactive}{[xshift=-\nodeShift,left=of camkiiPartial]}{CaMKII}{}
    \MacroMol{camkiiFull}{[xshift=\nodeShift,right=of camkiiPartial]}{CaMKII}{6P}

    \ProcessArrowWithShift{slowact}{camkiiInactive.east}{camkiiPartial.west}{2mm}
    \ProcessArrowWithShift{fastact}{camkiiPartial.east}{camkiiFull.west}{2mm}

    \ProcessArrowWithShift{dephospho1}{camkiiPartial.west}{camkiiInactive.east}{-2mm}
    \ProcessArrowWithShift{dephospho2}{camkiiFull.west}{camkiiPartial.east}{-2mm}


    % Ca+CaM binding.
    \draw (ca) -- (caCaMBinding);
    \draw (cam) -- (caCaMBinding);
    \draw[output] (caCaMBinding) -- (actCAM);

    \draw[required stimulus] (actCAM) -- (slowact);
    \draw[stimulus] (actCAM) -- (fastact);

    \draw[catalyst] (camkiiPartial.north) to[bend right,in=90,out=75] (fastact);

    % dephosphorylation.
    \MacroMol{pp1}{[yshift=-2*\ydist,below=of fastact]}{PP1}{}

    \Connect{pp1_camkii6}{required stimulus}{pp1}{dephospho2}
    %\Connect{pp1_camkii1}{required stimulus}{pp1_camkii6}{dephospho1}
    \draw[required stimulus] (pp1_camkii6) -| (dephospho1);

    % I1 and I1P.
    \Process{procI1PP1Cplx}{[yshift=-0.5*\ydist,below=of pp1]}{ }
    \MacroMol{i1p}{[xshift=-\xdist,left=of procI1PP1Cplx]}{I1}{P};
    \Process{procI1Phospho}{[xshift=-0.5*\xdist,left=of i1p]}{}
    \MacroMol{i1}{[xshift=-0.5*\xdist,left=of procI1Phospho]}{I1}{};
    \MacroMol{pka}{[yshift=1.5*\ydist,above of=i1]}{PKA}{}

    \draw (i1) -- (procI1Phospho);
    \Connect{i1phospo}{output}{procI1Phospho}{i1p}

    \Connect{wirePkaI1Phospho}{catalyst}{pka}{procI1Phospho}

    % Make complex of I1P and PP1.
    \MacroMol{pp1a}{[yshift=-2*\ydist,below=of pp1]}{PP1}{}
    \MacroMol{i1pa}{[right=of pp1a]}{I1}{P}
    \ComplexTwo{cplxI1PP1}{pp1a}{i1pa}{}

    \Connect{wireI1}{}{i1p}{procI1PP1Cplx}
    \Connect{wirePP1}{}{pp1}{procI1PP1Cplx}
    \Connect{wireCplx}{output}{procI1PP1Cplx}{cplxI1PP1}

    %%%%%%%%%%%%%%%%%%%%%%%%%%%%%%%%%%%%%%%%%%%%%%%%%%%%%%%%%%%%%%%%%%%%%%%%%%%
    %% Exchange of I1
    \node[rectangle,thick,draw
        ,minimum width=2cm,minimum height=1.5cm,label=-90:Cytosol] 
        (cytosol) [yshift=-\ydist,below=of i1] {};

    \MacroMol{i1Cyt}{at (cytosol)}{I1}{}

    \Process{procExchange}{[yshift=-0.5*\ydist,below=of i1]}{ }
    \Connect{x1}{ }{i1Cyt}{procExchange}
    \Connect{x2}{output}{procExchange}{i1}



\end{tikzpicture}    

\end{document}
