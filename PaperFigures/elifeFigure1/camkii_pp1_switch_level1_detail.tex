%%=====================================================================================
%%
%%    Description:  Level 1 details for  pp1 and camkii interaction
%%
%%        Version:  1.0
%%        Created:  07/11/2016
%%       Revision:  none
%%
%%         Author:  Dilawar Singh (), dilawars@ncbs.res.in
%%   Organization:  NCBS Bangalore
%%      Copyright:  Copyright (c) 2016, Dilawar Singh
%%
%%          Notes:  
%%                
%%=====================================================================================
\RequirePackage{luatex85,shellesc}
\documentclass[crop,tikz,class=../elife]{standalone}
\usetikzlibrary{fit,shapes,positioning,arrows}
\usetikzlibrary{arrows.meta}
\usetikzlibrary{calc}
\usepackage{xstring}

% \usepackage[sfdefault]{FiraSans}
\renewcommand\familydefault{\sfdefault}

% \CAMKIIRING{label}{x}{y}{list_of_phosphorylated_subunits}{radius_of_subunit}{no_of_subunits}



\newcommand{\tstar}[5]{% x, y, inner radius, outer radius, tips,
    \pgfmathsetmacro{\starangle}{360/#5}
    \draw[draw=none,fill=blue!40] [xshift=#1cm,yshift=#2cm](0:#3)
    \foreach \x in {1,...,#5}
    { -- (\starangle/2+\x*\starangle-\starangle/2:#4) -- (#4+\x*\starangle:#3)
    }
    -- cycle;
}

\newcommand{\TSTAR}[3]{% pos, radius, tips,
    \node[star,star points=#3,star point ratio=0.3,fill=blue!40,minimum
    size=#2cm,] (inner#3) at (#1) {};
}


\newcommand{\CAMKIIRING}[6] { % name, x, y, indices_of_red_balls, size, tips 
    % get the theta for given number of subunits.
    \begin{scope}[ ]
        \pgfmathsetmacro{\theta}{360/#6};
        \pgfmathsetmacro{\r}{0.5*#5/sin(\theta/2)};
        \def\fitlist{};
        \foreach \i in {1,...,#6}
        {
            \IfSubStr {#4} {\i}
            {
                \node[ball color=red,circle,shading=ball,minimum size=#5cm] (r\i) at
                ($(\i*\theta:\r cm)+(#2,#3)$) {};
            }
            {
                \node[draw=none,ball color=blue,circle,minimum size=#5cm](r\i) at
                ($(\i*\theta:\r cm)+(#2,#3)$) {}; 
            }
            \xdef\fitlist{\fitlist(r\i)};
        };
        % inner subunit.

        \tstar{#2}{#3}{\r/5}{\r}{#6};
        \node[circle,fit=\fitlist,] (#1) {};
        %\node[] at (#2,#3) {#1};
    \end{scope}
}

\newcommand{\CAMKII}[5] { % name, position, indices_of_red_balls, size, tips 
    % get the theta for given number of subunits.
    \begin{scope}[ ]
        \pgfmathsetmacro{\theta}{360/#5};
        \pgfmathsetmacro{\r}{0.5*#4/sin(\theta/2)};
        \def\fitlist{};

        \node[draw=none,#2] (#1_root) {};

        \foreach \i in {1,...,#5}
        {
            \IfSubStr {#3} {\i}
            {
                \node[draw=none,inner sep=0,ball color=red,circle,minimum size=#4cm] (r\i) at
                ($(\i*\theta:\r cm)+(#1_root)$) {};
            }
            {
                \node[draw=none,ball color=blue,inner sep=0,circle,minimum size=#4cm](r\i) at
                ($(\i*\theta:\r cm)+(#1_root)$) {}; 
            }
            \xdef\fitlist{\fitlist(r\i)};
        };
        % inner subunit.
        \TSTAR{#1_root.center}{0.5*\r}{#5};
        \node[fit=\fitlist,circle,inner sep=0] (#1) {};
    \end{scope}
}

\newcommand{\CACAM}[3] { % name, (x, y), size 
    \begin{scope}[]
        \pgfmathsetmacro{\car}{#3/2}
        \node[ball color=green,circle,minimum size=#3 cm,inner sep=0pt] 
            (rcam) at #2 {};

        \foreach \x in {1,...,4} 
        {
            \pgfmathsetmacro{\theta}{360/4*\x};
            \node[inner sep=0pt,ball color=yellow,circle,minimum size=\car cm] 
            (ca\x) at ($#2+(\theta:\car cm)$) {};
        }
        \node[circle,fit=(ca1) (ca2) (ca3) (ca4) (rcam)] (#1)  {};
    \end{scope}

}



%%%%%%%%%%%%%%%%%%%%%%%%%%%%%%
%% Chemical equilibrium arrow 

\newdimen\arrowsize
\newdimen\mylw
\pgfkeys{/my arrows/chemeq/.style={draw,thick,double distance=3pt,onearc-onearc}}
\pgfkeys{/my arrows/size/.code={\pgfsetarrowoptions{onearc}{#1}}}
\def\myalw{1pt}
\pgfarrowsdeclare{onearc}{onearc}{%
  \mylw=\myalw
  \pgfarrowsleftextend{-\pgfgetarrowoptions{onearc}-.5\mylw}
  \pgfarrowsrightextend{1pt}
}{%
  \pgfsetdash{}{0pt}
  \mylw=\pgflinewidth
  \pgfsetlinewidth{\myalw}
  \advance\arrowsize by.5\pgflinewidth
  \pgfpathmoveto{\pgfpoint{-\pgfgetarrowoptions{onearc}}{-\pgfgetarrowoptions{onearc}-.5\mylw}}%
  \pgfpatharc{180}{90}{\pgfgetarrowoptions{onearc}}
  \pgfusepathqstroke
}


\begin{document}

% New key: shorten both side.
\tikzset{ shorten <>/.style={ shorten >=#1, shorten <=#1 } }
\tikzset{ every label/.style={font=\Large}}

\centering
\begin{tikzpicture} [ node distance=3cm
        , Species/.style = { inner sep=1pt, circle, shading=ball, ball color=green!80 }
        , PP1/.style = { star, star points=2, minimum size = 1cm }
        , I1P/.style = { rectangle, fill=blue, minimum size = 0.3cm }
        , CaM/.style = { star,  minimum size=1.5cm, star points = 4
            , star point ratio=2 }
        , every path/.style = { ->, thick}
        , Process/.style = { circle, inner sep=1pt, minimum size=1pt, fill=black }
    ]


    % In this scope, calcium binds to Calmodulin.
    \begin{scope}
        \node[CaM, fill=blue!50, label=above:CaM] (cam) {};
        \node[CaM, fill=blue!80, right=of cam, label=above:Ca-CaM] (cacam) {};
        \foreach \i in {1,2,3,4} {
            \node[Species, inner sep=1pt] (ca\i) at (cacam.outer point \i) {};
        }

        %% Calcium comes from here
        \node[rectangle, fit=(cacam) (ca1) (ca2) (ca3) (ca4)] (cacamComplex) {};

        \node[inner sep=0pt] (caCamBinding) at ($(cam)!0.5!(cacam)$) {};

        \draw[->, line width=3pt] (cam) to[] node[] (binding) {} (cacamComplex);
        %% Draw an incoming calcium
        \node[Species, label=Ca] (incomingCa) at ([yshift=1cm,xshift=-1cm]binding) {};
        \draw[-.] (incomingCa) to[,bend right] (binding.center);
    \end{scope}

    %% In this scope, CaCaM activates CaMKII.
    \begin{scope}

        \CAMKII{camkii0}{below=of ca1.center}{0}{0.5}{6};
        \CAMKII{camkii1}{right=of camkii0}{1}{0.5}{6};
        \CAMKII{camkii11}{right=of camkii1}{1,2,3,4,5,6}{0.5}{6};

        \draw[->, shorten <>= 0.1cm] (camkii0) 
            edge  node[Process] (phospho1) {} (camkii1);


        \draw[->, shorten <>= 0.1cm, line width=2.5pt] (camkii1) 
            to[] node[Process] (phospho2) {} (camkii11);

        % Autophosphorylation.
        \draw[-*] (camkii11.north west) to[bend right] (phospho2.north);

        \draw[-.,bend right ] (cacamComplex) to[] (phospho1.center); 

        
    \end{scope}

    % Dephosphorylation of CaMKII
    \begin{scope}

        \draw[->, bend left ] (camkii11) 
            to[] node[Process] (dephosphorylation) {} (camkii0);
        
        \node[I1P, fill=blue!80, label=below:I1P] (i1p) [below of=camkii11] {};
        \node[PP1, fill=blue!80, label=below:PP1] (pp1) [left of=i1p] {};

        % PP1 turns active camkii to inactive camkii
        \draw[-.] (pp1.north) to[ bend right] (dephosphorylation);
        
        % I1P inhibits PP1
        \draw[-|, line width=2pt] (i1p) -- (pp1);

    \end{scope}

\end{tikzpicture}


\end{document}

