%=====================================================================================
%    Description:  Multiple bistables in PSD synchronized by subunit exchange.
%    87mm version.
%=====================================================================================
\RequirePackage{luatex85,shellesc}
\documentclass[crop=true,multi=false,varwidth=15.0cm,class=../elife]{standalone}
\usepackage{pgfplots}
\usepackage{xstring}
\usepackage{xcolor}
\usepackage{siunitx}
\usepackage{tabularx}

% Cokumntype P has fixed width and align to left as well. See 
% https://tex.stackexchange.com/a/7348/8087 
\usepackage{array}
\usepackage{ragged2e}
\newcolumntype{P}[1]{>{\raggedright\arraybackslash}p{#1}}
% \renewcommand\familydefault{\sfdefault}
\pgfplotsset{compat=1.15}
\usetikzlibrary{calc,positioning}
\usetikzlibrary{shapes,arrows,arrows.meta,fit}

\usepgfplotslibrary{patchplots}
\usepgfplotslibrary{units}
\usepackage{xifthen} % provides isempty
% \CAMKIIRING{label}{x}{y}{list_of_phosphorylated_subunits}{radius_of_subunit}{no_of_subunits}


\newcommand{\tstar}[5]{% x, y, inner radius, outer radius, tips,
    \pgfmathsetmacro{\starangle}{360/#5}
    \draw[draw=none,fill=blue!40] [xshift=#1cm,yshift=#2cm](0:#3)
    \foreach \x in {1,...,#5}
    { -- (\starangle/2+\x*\starangle-\starangle/2:#4) -- (#4+\x*\starangle:#3)
    }
    -- cycle;
}

\newcommand{\TSTAR}[3]{% pos, radius, tips,
    \node[star,star points=#3,star point ratio=0.3,fill=blue!40,minimum
    size=#2cm,] (inner#3) at (#1) {};
}


\newcommand{\CAMKIIRING}[6] { % name, x, y, indices_of_red_balls, size, tips 
    % get the theta for given number of subunits.
    \pgfmathsetmacro{\theta}{360/#6};
    \pgfmathsetmacro{\r}{0.5*#5/sin(\theta/2)};
    \tstar{#2}{#3}{\r/5}{\r}{#6};
    \begin{scope}[ ]
        \def\fitlist{};
        \foreach \i in {1,...,#6}
        {
            \IfSubStr {#4} {\i}
            {
                \node[ball color=red,circle,shading=ball,minimum size=#5cm] (r\i) at
                ($(\i*\theta:\r cm)+(#2,#3)$) {};
            }
            {
                \node[draw=none,ball color=blue,circle,minimum size=#5cm](r\i) at
                ($(\i*\theta:\r cm)+(#2,#3)$) {}; 
            }
            \xdef\fitlist{\fitlist(r\i)};
        };
        % inner subunit.

        \node[circle,fit=\fitlist,] (#1) {};
        %\node[] at (#2,#3) {#1};
    \end{scope}
}

\newcommand\SUBUNIT[4]{ % name, pos, radius, color
    \node[draw=none,ball color=#4,inner sep=0,circle,minimum size=#3 cm] (#1) at #2 {}; 
}

\newcommand{\CAMKII}[5] { % name, position, indices_of_red_balls, size, tips 
    % get the theta for given number of subunits.
    \node (#1) at (#2) {};
    \begin{scope}[ ]
        \pgfmathsetmacro{\theta}{360/#5};
        \pgfmathsetmacro{\r}{0.5*#4/sin(\theta/2)};
        \def\fitlist{};

        \node[draw=none] (#1_root) at (#1) {};
        % inner subunit.
        \TSTAR{#1_root.center}{0.4*\r}{#5};

        \foreach \i in {1,...,#5}
        {
            \IfSubStr {#3} {\i}
            {
                \node[draw=none,inner sep=0,ball color=red,circle,minimum size=#4cm] (r\i) at
                ($(\i*\theta:\r cm)+(#1_root)$) {};
            }
            {
                \node[draw=none,ball color=blue,inner sep=0,circle,minimum size=#4cm](r\i) at
                ($(\i*\theta:\r cm)+(#1_root)$) {}; 
            }
            \xdef\fitlist{\fitlist(r\i)};
        };
        \node[fit=\fitlist,circle,inner sep=0] (#1) {};
    \end{scope}
}

\newcommand{\CAMKIIWithoutHub}[5] { % name, position, indices_of_red_balls, size, tips 
    % get the theta for given number of subunits.
    \node (#1) at (#2) {};
    \begin{scope}[ ]
        \pgfmathsetmacro{\theta}{360/#5};
        \pgfmathsetmacro{\r}{0.5*#4/sin(\theta/2)};
        \def\fitlist{};

        \node[draw=none] (#1_root) at (#1) {};

        \foreach \i in {1,...,#5}
        {
            \IfSubStr {#3} {\i}
            {
                \node[draw=none,inner sep=0,ball color=red,circle,minimum size=#4cm] (r\i) at
                ($(\i*\theta:\r cm)+(#1_root)$) {};
            }
            {
                \node[draw=none,ball color=blue,inner sep=0,circle,minimum size=#4cm](r\i) at
                ($(\i*\theta:\r cm)+(#1_root)$) {}; 
            }
            \xdef\fitlist{\fitlist(r\i)};
        };
        \node[fit=\fitlist,circle,inner sep=0] (#1) {};
    \end{scope}
}


\newcommand{\CACAM}[3] { % name, (x, y), size 
    % A node is created with given name which fit all others.
    \node (#1) at (#2) { };
    \edef\name{#1_center}
    \begin{scope}[]
        \pgfmathsetmacro{\car}{#3/2}
        \node[star,star points=4,fill=red,minimum size=#3 cm,inner sep=0pt] 
            (\name) at (#1) {};

        \foreach \x in {1,...,4} 
        {
            \pgfmathsetmacro{\theta}{360/4*\x};
            \node[inner sep=0pt,ball color=yellow,circle,minimum size=\car cm] 
            (ca\x) at ($(#2)+(\theta:\car cm)$) {};
        }
        \node[circle,inner sep=0,fit=(ca1) (ca2) (ca3) (ca4) (\name)] (#1)  {};
    \end{scope}
}

\newcommand{\CAM}[3] { % name, (x, y), size 
    \node (#1) at (#2) { };
    \begin{scope}[]
        \pgfmathsetmacro{\car}{#3/2}
        \node[star,star points=4,star point ratio=2
            , fill=red,minimum size=#3 cm,inner sep=0pt] 
            (rcam) at (#1) {};
    \end{scope}
}

% NMDA receptors and other.
% \NMDAOLD}{(x,y)}{width}{gap}{rotation}
\newcommand{\NMDAOLD}[5] {
    \pgfmathsetmacro\rwidth{#3/5.0}
    \pgfmathsetmacro\gap{0.1+#4}

    \edef\ang{#5}
    \node[minimum height=#3 cm,minimum width=\gap cm] (#1) at #2 {};
    \begin{scope}[rotate around={(\ang:#2)}]
        \foreach \xshift/\name in {\gap/#1_right,-\gap/#1_left}
        {
            \node[draw=blue,fill=red!20,rectangle, inner sep=0pt
                , decorate, decoration={random steps,amplitude=1pt,segment length=1pt}
                , minimum height=#3cm ,minimum width=\rwidth cm
            ] (\name) at ([xshift=\xshift cm]#1.east) {};
        }
    \end{scope}
}

% \AMPA{name}{(x0,y0)}{height}{gap or opening size}{rotation}
\newcommand{\AMPA}[5] {
    \pgfmathsetmacro{\rwidth}{#3/5.0}
    \pgfmathsetmacro\rwidth{#3/5.0}
    \pgfmathsetmacro\gap{0.1+#4}
    \edef\ang{#5}

    \node (#1) at #2 {};
    \begin{scope}[rotate around={(\ang:#2)}]
        \foreach \i in {\gap,0}
        {
            \node[minimum height=#3 cm,minimum width=\gap cm
                ,transform shape   % necessary when within scope
                ,cylinder,fill=red
            ] (#1_\i) at ([yshift=\i cm]#1) {};
        }
    \end{scope}
}

% \CA{label}{coordinate}{label}
\newcommand{\CA}[3] {
    \node (#1) at #2 {};
    \node[shading=ball,circle,ball color=yellow,inner sep=0,minimum size=2 mm]
        at (#1) { };
}


%%%%%%%%%%%%%%%%%%%%%%%%%%%%%%
%% Chemical equilibrium arrow 

\newdimen\arrowsize
\newdimen\mylw
\pgfkeys{/my arrows/chemeq/.style={draw,thick,double distance=3pt,onearc-onearc}}
\pgfkeys{/my arrows/size/.code={\pgfsetarrowoptions{onearc}{#1}}}
\def\myalw{1pt}
\pgfarrowsdeclare{onearc}{onearc}{%
  \mylw=\myalw
  \pgfarrowsleftextend{-\pgfgetarrowoptions{onearc}-.5\mylw}
  \pgfarrowsrightextend{1pt}
}{%
  \pgfsetdash{}{0pt}
  \mylw=\pgflinewidth
  \pgfsetlinewidth{\myalw}
  \advance\arrowsize by.5\pgflinewidth
  \pgfpathmoveto{\pgfpoint{-\pgfgetarrowoptions{onearc}}{-\pgfgetarrowoptions{onearc}-.5\mylw}}%
  \pgfpatharc{180}{90}{\pgfgetarrowoptions{onearc}}
  \pgfusepathqstroke
}


%  \PHOSPHO{name}{location}
\newcommand\PHOSPHO[2]
{
    \node[circle,fill=yellow,inner sep=0pt] (#1) at (#2) {\tiny P};
}


\newcommand{\CAMKIIHOLOENZYME}[6] { % name, x, y, indices_of_red_balls, size, tips 
    % get the theta for given number of subunits.
    \begin{scope}[ ]
        \pgfmathsetmacro{\theta}{360/#6};
        \pgfmathsetmacro{\r}{0.5*#5/sin(\theta/2)};
        \def\fitlist{};
        \foreach \i in {1,...,#6}
        {
            \IfSubStr {#4} {\i}
            {
                \node[ball color=red,circle,shading=ball,minimum size=#5cm] (r\i) at
                ($(\i*\theta:\r cm)+(#2,#3,0)$) {};
            }
            {
                \node[draw=none,ball color=blue,circle,minimum size=#5cm](r\i) at
                ($(\i*\theta:\r cm)+(#2,#3,0)$) {}; 
            }
            \xdef\fitlist{\fitlist(r\i)};
        };

        % inner hub.
        % \tstar{#2}{#3}{\r/5}{\r}{#6};

        \foreach \i in {1,...,#6}
        {
            \IfSubStr {#4} {\i}
            {
                \node[ball color=red,circle,shading=ball,minimum size=#5cm] (r\i) at
                ($(\i*\theta:\r cm)+(#2,#3,\r)$) {};
            }
            {
                \node[draw=none,ball color=blue,circle,minimum size=#5cm](r\i) at
                ($(\i*\theta:\r cm)+(#2,#3,\r)$) {}; 
            }
            \xdef\fitlist{\fitlist(r\i)};
        };
        \node[circle,fit=\fitlist,] (#1) {};
        %\node[] at (#2,#3) {#1};
    \end{scope}
}



\begin{document}

\pgfmathsetmacro{\figW}{15.0}
\pgfmathsetmacro{\colW}{0.45*\figW}
\pgfmathsetmacro{\plotW}{0.9*\colW}
\pgfmathsetmacro{\plotH}{2.5}
\def\PSDLENGTH{\plotW}
\pgfplotsset{axis lines=left}
\pgfplotsset{legend style={fill=none,draw=none}}

% Used in last row to control height of surf plots.
\edef\HEIGHT{40mm}

\pgfplotsset{
    , xtick align=center
    , ytick align=center
    , unit markings=parenthesis
    , xticklabel style={/pgf/number format/fixed}
    , axis line style={-}
    , ylabel near ticks
    , xlabel near ticks
}

% label macro
\newcommand\LABEL[2]{\node[above left=of #1.north west,xshift=0mm,yshift=-8mm]{\bf #2};}
\newcommand\SUB[2]{#1\textsubscript{#2}}

% macro for plotting trajectories and its histogram.
\newcommand\PlotTrajWithHist[6]{ %name,location,filename,every,color,extra
    \edef\plotH{1.5}
    \begin{axis}[name=#1, at={#2}, anchor=north west
        , width=0.65*\plotW cm, height=\plotH cm,scale only axis 
        , ytick={0,6,12,18}
        , enlarge y limits=0.1
        , enlarge x limits=0.1
        , ylabel={Active CaMKII}
        , ylabel style={font=\footnotesize}
        , title style={at={(0.5,0.75)}}
        , #6
        ]
        \addplot [color=#5] gnuplot [ raw gnuplot, id=#1fig21 ] { 
            plot "#3" every #4 using (column("time")/86400):"CaMKII*"; 
        };
    \end{axis}
    \begin{axis}[
        , enlarge y limits=0.1
        , enlarge x limits=0.1
        , height=\plotH cm, width=0.8*\plotH cm,scale only axis
        , ytick={0,6,12,18}
        , ymax=18, xmax=0.025
        , yticklabels=\empty
        , at = (#1.north east), xshift=3mm
        , anchor = north west
        , x filter/.code=\pgfmathparse{rawy}
        , y filter/.code=\pgfmathparse{rawx}
        , try min ticks=2
        , scaled ticks=false
        , title style={font=\scriptsize,at={(0.5,0.75)}}
        , ymin = 0
        ]
        \addplot [fill=#5,xbar, draw=#5, bar width=0.5mm] gnuplot [ raw gnuplot,id=#1fig ] 
        {
            stats "#3" using "CaMKII*" prefix "A";
            bin(x,width)=width*floor(x/width);
            plot "#3" using (bin(column("CaMKII*"),1)):(1.0/A_sum) smooth freq;
        };
    \end{axis}
}

% Controls the stretch in which free subunits are drawn.
\pgfmathsetmacro{\fW}{0.95*\plotW}
% Controls the size of subunits.
\edef\SUSIZE{0.08}
\edef\CylWidth{15mm}

\begin{tabularx}{\figW cm}{X X}
    \begin{tikzpicture}[baseline]
        % Draw camkii rings.
        \foreach \i/\x in {-3/1,0/2,3/3}
        {
            \node[ draw=red, dotted, thick, circle, minimum size=\CylWidth
                , label={[node distance=3mm]above:{\small Cluster \x}}
                ] (switch\x)  at (\i,0) { };

            \foreach \j in {1,...,6}
            {
                \edef\activeSE{random(1,7),random(1,7),random(1,7),random(1,7),random(1,7),random(1,7),random(1,7)}
                % Each switch has some camkii here.
                \coordinate (camkii0) at (\i+0.4*rand,0.3*\CylWidth*rand);
                \begin{scope}[rotate=30*rand]
                    \pgfmathsetmacro{\NumberOfSuInRing}{random(6,7)}
                    \CAMKIIWithoutHub{c0}{camkii0}{\activeSE}{\SUSIZE}{\NumberOfSuInRing};
                \end{scope}
            }
        }

        % Draw PSD over switches.
        \node[cylinder, minimum width=\CylWidth, minimum height=0.8*\figW cm
            , draw=blue, fill=blue!50, opacity=0.1, inner sep=1mm
            ] (psd) at (switch2) {};

        % K distance away from each other.
        \draw[|<->|,thick] let \p{r1}=(switch1), \p{r2}=(switch2),
            \p{y0}=([yshift=-2mm]psd.south)
            in  (\x{r1},\y{y0}) -- (\x{r2},\y{y0}) node[below,midway] {d};

        \foreach \j in {1,2,...,12}
        {
            \node[shade,ball color=red, fill=red, circle
            , minimum size=\SUSIZE cm ,inner sep=1pt] (su\j) 
            at (\fW/2*rand,0.4*\CylWidth*rand) {};
            \pgfmathsetmacro{\Ang}{(rand>0)?0:180}
            \draw[-latex,color=red,very thin] (su\j) -- ++(\Ang:2mm);
        }

        \foreach \j in {1,2,...,9}
        {
            \node[shade,ball color=blue,fill=blue,circle
            ,minimum size=\SUSIZE cm, inner sep=1pt] 
            (su\j) at (\fW/2*rand,0.4*\CylWidth*rand) { };
            \pgfmathsetmacro{\Ang}{(rand>0)?0:180}
            \draw[-latex,color=blue,very thin] (su\j) -- ++(\Ang:2mm);
        }
        \node[xshift=0mm, yshift=-3mm, above=of psd.bottom] (labeld) {\bf A};
    \end{tikzpicture} 
    \\
   \begin{tikzpicture}[scale=1, every node/.style={} ]
        \PlotTrajWithHist{without_su_axis}{(0,0)}{%
            ./CaMKII-6+PP1-27+L-125e-9+N-3+diff-0.dat_processed.dat}{20}{red}{
                title={-SE,\SUB{D}{sub}=0},xlabel=Time, x unit=day
        }
        \LABEL{without_su_axis}{B}
    \end{tikzpicture}  
    &
   \begin{tikzpicture}[scale=1, every node/.style={} ]
        \PlotTrajWithHist{with_su_axis}{(without_su_axis.below south west)}{%
            ./CaMKII-6+PP1-27+L-125e-9+N-3+diff-1e-12.dat_processed.dat}{20}{blue}{%
            xlabel=Time, x unit=day
            , title={+SE,\;\SUB{D}{sub}=\SI{1}{\micro\meter\squared\per\second}}
            , yshift=-2mm
        }
    \end{tikzpicture}  
    \\
    \begin{tikzpicture}[scale=1]
        \begin{semilogxaxis}[name=f1
            , axis lines*=left
            , xlabel= \SUB{D}{sub}
            , x unit=\si{\micro\meter \squared \per \second}
            % , label style={font=\scriptsize}
            , height=\HEIGHT 
            , enlargelimits
            , legend style={fill=none,at={(1.01,0.5)},anchor=south west}
            , legend columns=1
            , ymax = 1, ymin = 0
            , ytick={0,1}
            , title={d=\SI{30}{\nano\meter}}
            ]
            \addplot+ [color=blue] table [
                x expr={\thisrow{diff const}*1e12}
                , y=intermediate states mean
                , col sep=comma]{decay_of_intermediate_states.dat};

            \addplot+ [color=red] table [
                x expr={\thisrow{diff const}*1e12}
                , y expr={1-\thisrow{intermediate states mean}}
                , col sep=comma]{decay_of_intermediate_states.dat};
            \legend{\SUB{t}{i}, \SUB{k}{s}=1-\SUB{t}{i}};
        \end{semilogxaxis}
        \node[above=of f1.north west, yshift=-5mm, xshift=-15mm] {\bf C};
    \end{tikzpicture} 
    &
    \begin{tikzpicture}[]
        \pgfplotsset{axis lines*=box}
        \begin{semilogxaxis}[name=f2
            , axis lines*=box
            , change y base=true
            , view={0}{90}
            , xlabel= \SUB{D}{sub}
            , x unit=\si{\micro\meter\squared\per\second}
            , title=\SUB{k}{s}
            , height=\HEIGHT 
            , colorbar
            , colorbar style={width=1.5mm}
            , enlargelimits=false
            , ylabel = d, y unit=\si{\meter}
            , y SI prefix=nano % scales the axis appropriately unlike \si{nm}.
            , ylabel shift=-2mm 
            % , label style={font=\scriptsize}
            ]
            \addplot3 [surf,mesh/rows=10 ] table[col sep=comma
            , x expr={\thisrow{D}*1e12} % unit are um^2/sec
            , z expr={1-\thisrow{intermediate_state}}]{./data_sync_phase_plot.csv};
        \end{semilogxaxis}
        \node[above=of f1.north west, yshift=-5mm, xshift=-15mm] {\bf D};
    \end{tikzpicture}
\end{tabularx}
\end{document}
