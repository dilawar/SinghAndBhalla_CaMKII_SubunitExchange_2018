% lualatex --shell-escape 
\RequirePackage{luatex85}
\documentclass[crop,tikz]{standalone}
\usepackage{pgfplots}
\usetikzlibrary{calc,graphs,graphdrawing,fit,positioning}
\usetikzlibrary{decorations,decorations.footprints,decorations.shapes}
\usetikzlibrary{shapes,arrows,arrows.meta}
\usegdlibrary{layered}
\usepackage{xstring}
\pgfplotsset{compat=1.15}

\begin{document}

\pgfmathsetseed{10}


\newcommand{\tstar}[5]{% x, y, inner radius, outer radius, tips,
    \pgfmathsetmacro{\starangle}{360/#5}
    \draw[draw=none,fill=blue!40] [xshift=#1cm,yshift=#2cm](0:#3)
    \foreach \x in {1,...,#5}
    { -- (\starangle/2+\x*\starangle-\starangle/2:#4) -- (#4+\x*\starangle:#3)
    }
    -- cycle;
}

\newcommand{\TSTAR}[3]{% pos, radius, tips,
    \node[star,star points=#3,star point ratio=0.3,fill=blue!40,minimum
    size=#2cm,] (inner#3) at (#1) {};
}


\newcommand{\CAMKIIRING}[6] { % name, x, y, indices_of_red_balls, size, tips 
    % get the theta for given number of subunits.
    \begin{scope}[ ]
        \pgfmathsetmacro{\theta}{360/#6};
        \pgfmathsetmacro{\r}{0.5*#5/sin(\theta/2)};
        \def\fitlist{};
        \foreach \i in {1,...,#6}
        {
            \IfSubStr {#4} {\i}
            {
                \node[ball color=red,circle,shading=ball,minimum size=#5cm] (r\i) at
                ($(\i*\theta:\r cm)+(#2,#3)$) {};
            }
            {
                \node[draw=none,ball color=blue,circle,minimum size=#5cm](r\i) at
                ($(\i*\theta:\r cm)+(#2,#3)$) {}; 
            }
            \xdef\fitlist{\fitlist(r\i)};
        };
        % inner subunit.

        \tstar{#2}{#3}{\r/5}{\r}{#6};
        \node[circle,fit=\fitlist,] (#1) {};
        %\node[] at (#2,#3) {#1};
    \end{scope}
}

\newcommand{\CAMKII}[5] { % name, position, indices_of_red_balls, size, tips 
    % get the theta for given number of subunits.
    \node (#1) at (#2) {};
    \begin{scope}[ ]
        \pgfmathsetmacro{\theta}{360/#5};
        \pgfmathsetmacro{\r}{0.5*#4/sin(\theta/2)};
        \def\fitlist{};

        \node[draw=none] (#1_root) at (#1) {};

        \foreach \i in {1,...,#5}
        {
            \IfSubStr {#3} {\i}
            {
                \node[draw=none,inner sep=0,ball color=red,circle,minimum size=#4cm] (r\i) at
                ($(\i*\theta:\r cm)+(#1_root)$) {};
            }
            {
                \node[draw=none,ball color=blue,inner sep=0,circle,minimum size=#4cm](r\i) at
                ($(\i*\theta:\r cm)+(#1_root)$) {}; 
            }
            \xdef\fitlist{\fitlist(r\i)};
        };
        % inner subunit.
        \TSTAR{#1_root.center}{0.5*\r}{#5};
        \node[fit=\fitlist,circle,inner sep=0] (#1) {};
    \end{scope}
}

\newcommand{\CACAM}[3] { % name, (x, y), size 
    % A node is created with given name which fit all others.
    \node (#1) at (#2) { };
    \edef\name{#1_center}
    \begin{scope}[]
        \pgfmathsetmacro{\car}{#3/2}
        \node[star,star points=4,fill=red,minimum size=#3 cm,inner sep=0pt] 
            (\name) at (#1) {};

        \foreach \x in {1,...,4} 
        {
            \pgfmathsetmacro{\theta}{360/4*\x};
            \node[inner sep=0pt,ball color=yellow,circle,minimum size=\car cm] 
            (ca\x) at ($(#2)+(\theta:\car cm)$) {};
        }
        \node[circle,inner sep=0,fit=(ca1) (ca2) (ca3) (ca4) (\name)] (#1)  {};
    \end{scope}
}

\newcommand{\CAM}[3] { % name, (x, y), size 
    \node (#1) at (#2) { };
    \begin{scope}[]
        \pgfmathsetmacro{\car}{#3/2}
        \node[star,star points=4,star point ratio=2
            , fill=red,minimum size=#3 cm,inner sep=0pt] 
            (rcam) at (#1) {};
    \end{scope}
}

% NMDA receptors and other.
% \NMDAOLD}{(x,y)}{width}{gap}{rotation}
\newcommand{\NMDAOLD}[5] {
    \pgfmathsetmacro\rwidth{#3/5.0}
    \pgfmathsetmacro\gap{0.1+#4}

    \edef\ang{#5}
    \node[minimum height=#3 cm,minimum width=\gap cm] (#1) at #2 {};
    \begin{scope}[rotate around={(\ang:#2)}]
        \foreach \xshift/\name in {\gap/#1_right,-\gap/#1_left}
        {
            \node[draw=blue,fill=red!20,rectangle, inner sep=0pt
                , decorate, decoration={random steps,amplitude=1pt,segment length=1pt}
                , minimum height=#3cm ,minimum width=\rwidth cm
            ] (\name) at ([xshift=\xshift cm]#1.east) {};
        }
    \end{scope}
}

% \AMPA{name}{(x0,y0)}{height}{gap or opening size}{rotation}
\newcommand{\AMPA}[5] {
    \pgfmathsetmacro{\rwidth}{#3/5.0}
    \pgfmathsetmacro\rwidth{#3/5.0}
    \pgfmathsetmacro\gap{0.1+#4}
    \edef\ang{#5}

    \node (#1) at #2 {};
    \begin{scope}[rotate around={(\ang:#2)}]
        \foreach \i in {\gap,0}
        {
            \node[minimum height=#3 cm,minimum width=\gap cm
                ,transform shape   % necessary when within scope
                ,cylinder,fill=red
            ] (#1_\i) at ([yshift=\i cm]#1) {};
        }
    \end{scope}
}

% \CA{label}{coordinate}{label}
\newcommand{\CA}[3] {
    \node (#1) at #2 {};
    \node[shading=ball,circle,ball color=yellow,inner sep=0,minimum size=2 mm]
        at (#1) { };
}


%%%%%%%%%%%%%%%%%%%%%%%%%%%%%%
%% Chemical equilibrium arrow 

\newdimen\arrowsize
\newdimen\mylw
\pgfkeys{/my arrows/chemeq/.style={draw,thick,double distance=3pt,onearc-onearc}}
\pgfkeys{/my arrows/size/.code={\pgfsetarrowoptions{onearc}{#1}}}
\def\myalw{1pt}
\pgfarrowsdeclare{onearc}{onearc}{%
  \mylw=\myalw
  \pgfarrowsleftextend{-\pgfgetarrowoptions{onearc}-.5\mylw}
  \pgfarrowsrightextend{1pt}
}{%
  \pgfsetdash{}{0pt}
  \mylw=\pgflinewidth
  \pgfsetlinewidth{\myalw}
  \advance\arrowsize by.5\pgflinewidth
  \pgfpathmoveto{\pgfpoint{-\pgfgetarrowoptions{onearc}}{-\pgfgetarrowoptions{onearc}-.5\mylw}}%
  \pgfpatharc{180}{90}{\pgfgetarrowoptions{onearc}}
  \pgfusepathqstroke
}


%  \PHOSPHO{name}{location}
\newcommand\PHOSPHO[2]
{
    \node[circle,fill=yellow,inner sep=0pt] (#1) at (#2) {\tiny P};
}



\begin{tikzpicture}[scale=1, every node/.style={},% node distance=5mm
    ]

    % Grid 
    %\node (origin) at (0,0) {+};
    \draw[thin,step=1,gray!10] (-10,-10) grid (5,5);

    % Draw synapse.
    \draw[color=blue,very thick] plot[smooth] coordinates { 
            (-8,-8) (-5,-8.2) (-3, -8)
            (-3, -4) (-7, -2) (-7,1) 
            (0, 1) (3,1) (2.5,-3) 
            (-1,-4) (-1,-8) (3,-8)
        };


    % \draw[blue,very thick] plot[smooth] coordinates { 
    %         (-8, -15) (-5, -15.5) (0,-15.1) (3, -15)
    %     };

    % draw PSD.
    \path[fill=blue!50,decorate,decoration={random steps}]
        (-6,1.2) -- (-5,1.2) -- (-4,1.2) -- (0,1.1) -- (3,1.2) 
        --  (1,0) -- (0,0) -- (-3,0) -- (-5,0) -- cycle;

    %%%%%%%%%%%%%%%%%%%%%%%%%%%%%%%%%%%%%%%%%%%%%%%%%%%%%%%%%%%%%%%%%%%%%%%%%%
    % A lot of calcium outside  and calmodulin inside.
    %%%%%%%%%%%%%%%%%%%%%%%%%%%%%%%%%%%%%%%%%%%%%%%%%%%%%%%%%%%%%%%%%%%%%%%%%%
    \foreach \i/\x/\y in {1/-7/2,2/-6.5/2.1,3/-6.2/1.9,4/-6.8/2.2}
    {
        \CA{ca\i}{(\x,\y)}{};
    }



    %%%%%%%%%%%%%%%%%%%%%%%%%%%%%%%%%%%%%%%%%%%%%%%%%%%%%%%%%%%%%%%%%%%%%%%%%%
    % Draw channels.
    % VDCC or voltage dependant calcium channel opens when membrane potential at
    % spine goes above threshold voltage of channel.
    %%%%%%%%%%%%%%%%%%%%%%%%%%%%%%%%%%%%%%%%%%%%%%%%%%%%%%%%%%%%%%%%%%%%%%%%%%
    \coordinate (vdcc) at (-6,1);
    \node[draw=blue,rotate=110,cylinder,fill=white,label=VDCC,minimum height=7 mm] 
        at (vdcc) {};

    \NMDAOLD{n0}{(-5,1)}{1}{0.05}{-10};

    \CA{canmda}{(n0.east)}{};

    \NMDAOLD{n1}{(1,1)}{1}{0.1}{10};

    % Attach a CaMKII below it.
\CAMKII{camkiiPSD}{[shift=(-135:5 mm)]n1}{1,2,3,4,5,6}{0.2}{6};
    
    \node[left=of n1,yshift=-5 mm,label=PP1,circle,fill=red] (pp1) {};
    \draw[-triangle 90 reversed] (pp1) edge[out=-0,in=180] (camkiiPSD);

    % CaMKII partial.
    \node[left=of camkiiPSD,shift=(60:-2cm)] (camkii) {};
    \CAMKII{camkiiPartial}{[xshift=0 cm]camkii}{1,2,5}{0.2}{6}

    %%%%%%%%%%%%%%%%%%%%%%%%%%%%%%%%%%%%%%%%%%%%%%%%%%%%%%%%%%%%%%%%%%%%%%%%%%%%
    % Calcium inflow  
    %%%%%%%%%%%%%%%%%%%%%%%%%%%%%%%%%%%%%%%%%%%%%%%%%%%%%%%%%%%%%%%%%%%%%%%%%%%%

    % Calcium can flow through VDCC
    \CA{ca0}{(vdcc)}{};
    \CACAM{cacam0}{-5,0}{0.3}
    \CAM{cam0}{-6.5,0}{0.3}

    \draw[-*,very thick] (cam0) to[midway] node (binding) {} (cacam0);
    \draw[->] (ca0) to[] (binding);

    \foreach \i in {1,2,...,5}
    {
        \coordinate (cacam) at (rnd*3-6,rnd-1)  {};
        %\CACAM{cacam\i}{cacam}{0.3}
    }


    %%%%%%%%%%%%%%%%%%%%%%%%%%%%%%%%%%%%%%%%%%%%%%%%%%%%%%%%%%%%%%%%%%%%%%
    % Channels 
    %%%%%%%%%%%%%%%%%%%%%%%%%%%%%%%%%%%%%%%%%%%%%%%%%%%%%%%%%%%%%%%%%%%%%%

    \node[below right=of cacam0] (can) {CaN};
    \node[below=of can] (i1p) {I1P/I2P};
    \node[above right=of i1p,xshift=5 mm] (ppx) {PP1/PP2};
    \node[left=of i1p,rotate=-90] (pka) {PKA};

     %%%%%%%%%%%%%%%%%%%%%%%%%%%%%%%%%%%%%%%%%%%%%%%%%%%%%%%%%%%%%%%%%%%%%%
     % Transportation of AMPA.
     %%%%%%%%%%%%%%%%%%%%%%%%%%%%%%%%%%%%%%%%%%%%%%%%%%%%%%%%%%%%%%%%%%%%%%

    \draw[->,decorate,decoration={shape backgrounds,shape=dart,shape size=2 mm}] 
        (camkiiPartial) -- (camkiiPSD);

    %\draw[-*] (ca0) edge[ ]  (cacam0);

    \draw[-*] (cacam0) edge[ ] (can);
    \draw[-triangle 90 reversed] (can) edge[ ] (i1p);
    \draw[-triangle 90 reversed] (i1p) edge[ ] (ppx);
    \draw[-triangle 90 reversed] (pka) edge[ ] (i1p);
    
    % Phosphorylation
    \draw[-*] (cacam0) edge[ ] node[midway] (phospho) {} (camkiiPartial);
    \draw[-latex] (camkiiPartial.north) edge[out=90,in=120] (phospho);
    \draw[-triangle 90 reversed] (ppx) edge[] (camkiiPartial);
    \draw[-triangle 90 reversed] (cacam0) edge[bend right] (pka.west);

    %% Transportation of AMPA channels.
    \AMPA{a0}{(2.8,-2)}{1}{0.1}{10};
    \CAMKII{camkiiCyt1}{[shift=(135:2 cm)]a0}{1,2,3,4,5,6,7}{0.2}{7}
    \draw[-triangle 90 reversed] (ppx) edge[] (camkiiCyt1);

    % CaMKII help translocating AMPA channels via Ras/MaP-K. 
    \AMPA{a1}{(-2,0.8)}{1}{0.15}{90};

    % CaMKII phosphorylated AMPA thus increasing their conductance.
    \CAMKII{camkiiCyt2}{[shift=(-25:0.7 cm)]n0}{1,2,3,4,5,6,7}{0.2}{7}

    \node at ([yshift=10 mm]a1) {};
    \AMPA{a2}{(2.8,1)}{1}{0.1}{45};
    \draw[-*] (camkiiCyt2) edge[bend right] (a1);
    
    \draw[->,very thick
            ,decorate, decoration={footprints,foot length=10 pt,stride length=15pt}
            ] ($(a0)+(0.5,.5)$)  to[bend right] node[below,sloped,midway,yshift=-5 mm] (transport) 
            {\small MAPK} ([xshift=3mm]a2);

    \draw[-*] (camkiiCyt1) -- (transport);


\end{tikzpicture}    
\end{document}
