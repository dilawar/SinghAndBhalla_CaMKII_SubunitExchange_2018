\RequirePackage{luatex85,shellesc}
\documentclass[tikz,multi=false]{standalone}
\usepackage{pgfplots}
\usepackage{ifthen}
\usepackage{xstring}
\usepackage[sfdefault]{FiraSans}
\usepackage{eulervm}
\usepackage{circuitikzgit}
\usepackage{pgfplots}
\usetikzlibrary{calc,graphs,graphdrawing,fit,positioning}
\usetikzlibrary{decorations,decorations.footprints,decorations.shapes}
\usetikzlibrary{shapes,arrows,arrows.meta}
\usetikzlibrary{arrows,arrows.meta}
\usetikzlibrary{calc,positioning,fit,shapes}
\usegdlibrary{layered}
\usepackage{xstring}
\pgfplotsset{compat=1.15}

\begin{document}

% \CAMKIIRING{label}{x}{y}{list_of_phosphorylated_subunits}{radius_of_subunit}{no_of_subunits}


\newcommand{\tstar}[5]{% x, y, inner radius, outer radius, tips,
    \pgfmathsetmacro{\starangle}{360/#5}
    \draw[draw=none,fill=blue!40] [xshift=#1cm,yshift=#2cm](0:#3)
    \foreach \x in {1,...,#5}
    { -- (\starangle/2+\x*\starangle-\starangle/2:#4) -- (#4+\x*\starangle:#3)
    }
    -- cycle;
}

\newcommand{\TSTAR}[3]{% pos, radius, tips,
    \node[star,star points=#3,star point ratio=0.3,fill=blue!40,minimum
    size=#2cm,] (inner#3) at (#1) {};
}


\newcommand{\CAMKIIRING}[6] { % name, x, y, indices_of_red_balls, size, tips 
    % get the theta for given number of subunits.
    \pgfmathsetmacro{\theta}{360/#6};
    \pgfmathsetmacro{\r}{0.5*#5/sin(\theta/2)};
    \tstar{#2}{#3}{\r/5}{\r}{#6};
    \begin{scope}[ ]
        \def\fitlist{};
        \foreach \i in {1,...,#6}
        {
            \IfSubStr {#4} {\i}
            {
                \node[ball color=red,circle,shading=ball,minimum size=#5cm] (r\i) at
                ($(\i*\theta:\r cm)+(#2,#3)$) {};
            }
            {
                \node[draw=none,ball color=blue,circle,minimum size=#5cm](r\i) at
                ($(\i*\theta:\r cm)+(#2,#3)$) {}; 
            }
            \xdef\fitlist{\fitlist(r\i)};
        };
        % inner subunit.

        \node[circle,fit=\fitlist,] (#1) {};
        %\node[] at (#2,#3) {#1};
    \end{scope}
}

\newcommand\SUBUNIT[4]{ % name, pos, radius, color
    \node[draw=none,ball color=#4,inner sep=0,circle,minimum size=#3 cm] (#1) at #2 {}; 
}

\newcommand{\CAMKII}[5] { % name, position, indices_of_red_balls, size, tips 
    % get the theta for given number of subunits.
    \node (#1) at (#2) {};
    \begin{scope}[ ]
        \pgfmathsetmacro{\theta}{360/#5};
        \pgfmathsetmacro{\r}{0.5*#4/sin(\theta/2)};
        \def\fitlist{};

        \node[draw=none] (#1_root) at (#1) {};
        % inner subunit.
        \TSTAR{#1_root.center}{0.4*\r}{#5};

        \foreach \i in {1,...,#5}
        {
            \IfSubStr {#3} {\i}
            {
                \node[draw=none,inner sep=0,ball color=red,circle,minimum size=#4cm] (r\i) at
                ($(\i*\theta:\r cm)+(#1_root)$) {};
            }
            {
                \node[draw=none,ball color=blue,inner sep=0,circle,minimum size=#4cm](r\i) at
                ($(\i*\theta:\r cm)+(#1_root)$) {}; 
            }
            \xdef\fitlist{\fitlist(r\i)};
        };
        \node[fit=\fitlist,circle,inner sep=0] (#1) {};
    \end{scope}
}

\newcommand{\CAMKIIWithoutHub}[5] { % name, position, indices_of_red_balls, size, tips 
    % get the theta for given number of subunits.
    \node (#1) at (#2) {};
    \begin{scope}[ ]
        \pgfmathsetmacro{\theta}{360/#5};
        \pgfmathsetmacro{\r}{0.5*#4/sin(\theta/2)};
        \def\fitlist{};

        \node[draw=none] (#1_root) at (#1) {};

        \foreach \i in {1,...,#5}
        {
            \IfSubStr {#3} {\i}
            {
                \node[draw=none,inner sep=0,ball color=red,circle,minimum size=#4cm] (r\i) at
                ($(\i*\theta:\r cm)+(#1_root)$) {};
            }
            {
                \node[draw=none,ball color=blue,inner sep=0,circle,minimum size=#4cm](r\i) at
                ($(\i*\theta:\r cm)+(#1_root)$) {}; 
            }
            \xdef\fitlist{\fitlist(r\i)};
        };
        \node[fit=\fitlist,circle,inner sep=0] (#1) {};
    \end{scope}
}


\newcommand{\CACAM}[3] { % name, (x, y), size 
    % A node is created with given name which fit all others.
    \node (#1) at (#2) { };
    \edef\name{#1_center}
    \begin{scope}[]
        \pgfmathsetmacro{\car}{#3/2}
        \node[star,star points=4,fill=red,minimum size=#3 cm,inner sep=0pt] 
            (\name) at (#1) {};

        \foreach \x in {1,...,4} 
        {
            \pgfmathsetmacro{\theta}{360/4*\x};
            \node[inner sep=0pt,ball color=yellow,circle,minimum size=\car cm] 
            (ca\x) at ($(#2)+(\theta:\car cm)$) {};
        }
        \node[circle,inner sep=0,fit=(ca1) (ca2) (ca3) (ca4) (\name)] (#1)  {};
    \end{scope}
}

\newcommand{\CAM}[3] { % name, (x, y), size 
    \node (#1) at (#2) { };
    \begin{scope}[]
        \pgfmathsetmacro{\car}{#3/2}
        \node[star,star points=4,star point ratio=2
            , fill=red,minimum size=#3 cm,inner sep=0pt] 
            (rcam) at (#1) {};
    \end{scope}
}

% NMDA receptors and other.
% \NMDAOLD}{(x,y)}{width}{gap}{rotation}
\newcommand{\NMDAOLD}[5] {
    \pgfmathsetmacro\rwidth{#3/5.0}
    \pgfmathsetmacro\gap{0.1+#4}

    \edef\ang{#5}
    \node[minimum height=#3 cm,minimum width=\gap cm] (#1) at #2 {};
    \begin{scope}[rotate around={(\ang:#2)}]
        \foreach \xshift/\name in {\gap/#1_right,-\gap/#1_left}
        {
            \node[draw=blue,fill=red!20,rectangle, inner sep=0pt
                , decorate, decoration={random steps,amplitude=1pt,segment length=1pt}
                , minimum height=#3cm ,minimum width=\rwidth cm
            ] (\name) at ([xshift=\xshift cm]#1.east) {};
        }
    \end{scope}
}

% \AMPA{name}{(x0,y0)}{height}{gap or opening size}{rotation}
\newcommand{\AMPA}[5] {
    \pgfmathsetmacro{\rwidth}{#3/5.0}
    \pgfmathsetmacro\rwidth{#3/5.0}
    \pgfmathsetmacro\gap{0.1+#4}
    \edef\ang{#5}

    \node (#1) at #2 {};
    \begin{scope}[rotate around={(\ang:#2)}]
        \foreach \i in {\gap,0}
        {
            \node[minimum height=#3 cm,minimum width=\gap cm
                ,transform shape   % necessary when within scope
                ,cylinder,fill=red
            ] (#1_\i) at ([yshift=\i cm]#1) {};
        }
    \end{scope}
}

% \CA{label}{coordinate}{label}
\newcommand{\CA}[3] {
    \node (#1) at #2 {};
    \node[shading=ball,circle,ball color=yellow,inner sep=0,minimum size=2 mm]
        at (#1) { };
}


%%%%%%%%%%%%%%%%%%%%%%%%%%%%%%
%% Chemical equilibrium arrow 

\newdimen\arrowsize
\newdimen\mylw
\pgfkeys{/my arrows/chemeq/.style={draw,thick,double distance=3pt,onearc-onearc}}
\pgfkeys{/my arrows/size/.code={\pgfsetarrowoptions{onearc}{#1}}}
\def\myalw{1pt}
\pgfarrowsdeclare{onearc}{onearc}{%
  \mylw=\myalw
  \pgfarrowsleftextend{-\pgfgetarrowoptions{onearc}-.5\mylw}
  \pgfarrowsrightextend{1pt}
}{%
  \pgfsetdash{}{0pt}
  \mylw=\pgflinewidth
  \pgfsetlinewidth{\myalw}
  \advance\arrowsize by.5\pgflinewidth
  \pgfpathmoveto{\pgfpoint{-\pgfgetarrowoptions{onearc}}{-\pgfgetarrowoptions{onearc}-.5\mylw}}%
  \pgfpatharc{180}{90}{\pgfgetarrowoptions{onearc}}
  \pgfusepathqstroke
}


%  \PHOSPHO{name}{location}
\newcommand\PHOSPHO[2]
{
    \node[circle,fill=yellow,inner sep=0pt] (#1) at (#2) {\tiny P};
}


\newcommand{\CAMKIIHOLOENZYME}[6] { % name, x, y, indices_of_red_balls, size, tips 
    % get the theta for given number of subunits.
    \begin{scope}[ ]
        \pgfmathsetmacro{\theta}{360/#6};
        \pgfmathsetmacro{\r}{0.5*#5/sin(\theta/2)};
        \def\fitlist{};
        \foreach \i in {1,...,#6}
        {
            \IfSubStr {#4} {\i}
            {
                \node[ball color=red,circle,shading=ball,minimum size=#5cm] (r\i) at
                ($(\i*\theta:\r cm)+(#2,#3,0)$) {};
            }
            {
                \node[draw=none,ball color=blue,circle,minimum size=#5cm](r\i) at
                ($(\i*\theta:\r cm)+(#2,#3,0)$) {}; 
            }
            \xdef\fitlist{\fitlist(r\i)};
        };

        % inner hub.
        % \tstar{#2}{#3}{\r/5}{\r}{#6};

        \foreach \i in {1,...,#6}
        {
            \IfSubStr {#4} {\i}
            {
                \node[ball color=red,circle,shading=ball,minimum size=#5cm] (r\i) at
                ($(\i*\theta:\r cm)+(#2,#3,\r)$) {};
            }
            {
                \node[draw=none,ball color=blue,circle,minimum size=#5cm](r\i) at
                ($(\i*\theta:\r cm)+(#2,#3,\r)$) {}; 
            }
            \xdef\fitlist{\fitlist(r\i)};
        };
        \node[circle,fit=\fitlist,] (#1) {};
        %\node[] at (#2,#3) {#1};
    \end{scope}
}



\newcommand\SWITCH[6]{ % name, x, y, width, rotation, on-off
    \edef\y{0.2}
    \ifthenelse{#6=1}{\edef\y{0.1}}{}
    \begin{scope}[xshift=-#2, yshift=-#3, rotate=#5]
        \draw[-Circle,line width=#4] (0 cm,0 cm) node (#1_a) {} -- (1 cm,0);
        \draw[Circle-,line width=#4] (1.5 cm,0) -- (2.5 cm,0) node (#1_b) {};
        \draw[line width=#4] (1 cm,\y cm) -- (1.5 cm,\y cm);
        \draw[line width=#4] (1.25 cm,\y cm) -- ++(0,0.2 cm);
    \end{scope}
}

%%%%%%%%%%%%%%%%%%%%%%%%%%%%%%
%% Chemical equilibrium arrow 

\newdimen\arrowsize
\newdimen\mylw
\pgfkeys{/my arrows/chemeq/.style={draw,thick,double distance=3pt,onearc-onearc}}
\pgfkeys{/my arrows/size/.code={\pgfsetarrowoptions{onearc}{#1}}}
\def\myalw{1pt}
\pgfarrowsdeclare{onearc}{onearc}{%
  \mylw=\myalw
  \pgfarrowsleftextend{-\pgfgetarrowoptions{onearc}-.5\mylw}
  \pgfarrowsrightextend{1pt}
}{%
  \pgfsetdash{}{0pt}
  \mylw=\pgflinewidth
  \pgfsetlinewidth{\myalw}
  \advance\arrowsize by.5\pgflinewidth
  \pgfpathmoveto{\pgfpoint{-\pgfgetarrowoptions{onearc}}{-\pgfgetarrowoptions{onearc}-.5\mylw}}%
  \pgfpatharc{180}{90}{\pgfgetarrowoptions{onearc}}
  \pgfusepathqstroke
}

\newcommand\PreSynSide[1]{%name
    % Draw pre synaptic side.
    % Draw synapse. Pre synaptic side.
    \begin{scope}[rotate=180, local bounding box=#1]
        \draw[color=red!30,very thick, fill=red!10] plot[smooth] coordinates { 
            (-3,-5) (-3,-4) (-6,-2) (-7,1) 
            (0,1) (3,1) (2.5,-2) (-1,-4) 
            (-1,-5) 
        };
    \end{scope}

    \begin{scope}[rotate=-90]
        % Pre synaptic side fill-in.
        \tikzset{every path/.style={line width=3pt,dashed}}
        \tikzset{pre/.style={-{o[width=5pt]},very thick}}

        \foreach[count=\i] \j in {-5,-4,...,5}
        {
            \draw[pre,color=blue!20] plot[smooth] coordinates { 
                (-5+0.1*rnd,1.4+0.1*\i) 
                (-4+0.1*rnd,1.15+0.15*\i)
                (-3+0.1*rnd,-1.5+0.5*\i)
                (-2+0.1*rnd,-2.5+0.75*\i)
                (-1+0.1*rnd,-3+0.85*\i)
                (0+0.1*rnd,-3.3+0.9*\i)
                (1+0.1*rnd,-3.4+0.9*\i)
            };
        }
    \end{scope}
}


\begin{tikzpicture}[scale=1]

    \node[rotate=0, scale=1] (rawSyn) at (-12,0) 
    {
        \begin{tikzpicture}[scale=1, rotate=0]
        \PreSynSide{a1}
        \begin{scope}[xshift=4cm, yshift=-2.8cm]

            % Draw synapse.
            \draw[color=red,very thick, fill=red!20] plot[smooth] coordinates { 
                (-3, -8) (-3, -4) (-7, -2) (-7,1) 
                (0, 1) (3,1) (2.5,-3) 
                (-1,-4) (-1,-8)
            };

            % draw PSD.
            % PSD
            \draw[draw=black, dashed, fill=red!30] plot[smooth] coordinates { 
                (-6.5,0) (-6.3,1.1) (0,1.05) (2,1) (2.5,1) (2,0) (-4,0.5) (-6.5,0)
            };
            %\path[fill=blue!50,,decoration={random steps}]
            %(-6,1.2) -- (-5,1.2) -- (-4,1.2) -- (0,1.1) -- (3,1.2) 
            %--  (1,0) -- (0,0) -- (-3,0) -- (-5,0) -- cycle;

            %%%%%%%%%%%%%%%%%%%%%%%%%%%%%%%%%%%%%%%%%%%%%%%%%%%%%%%%%%%%%%%%%%%%%%%%%%
            % A lot of calcium outside  and calmodulin inside.
            %%%%%%%%%%%%%%%%%%%%%%%%%%%%%%%%%%%%%%%%%%%%%%%%%%%%%%%%%%%%%%%%%%%%%%%%%%
            \foreach \i/\x/\y in {1/-7/2,2/-6.5/2.1,3/-6.2/1.9,4/-6.8/2.2}
            {
                \CA{ca\i}{(\x,\y)}{};
            }

            %%%%%%%%%%%%%%%%%%%%%%%%%%%%%%%%%%%%%%%%%%%%%%%%%%%%%%%%%%%%%%%%%%%%%%%%%%
            % Draw channels.
            % VDCC or voltage dependant calcium channel opens when membrane potential at
            % spine goes above threshold voltage of channel.
            %%%%%%%%%%%%%%%%%%%%%%%%%%%%%%%%%%%%%%%%%%%%%%%%%%%%%%%%%%%%%%%%%%%%%%%%%%
            \coordinate (vdcc) at (-6,1);
            \node[draw=blue,rotate=110,cylinder,fill=white,label=VDCC,minimum height=7 mm] 
                at (vdcc) {};

            \NMDAOLD{n0}{(-5,1)}{1}{0.05}{-10};

            \CA{canmda}{(n0.east)}{};

            \NMDAOLD{n1}{(1,1)}{1}{0.1}{10};

            % Attach a CaMKII below it.
            \CAMKII{camkiiPSD}{left=0mm of n1.south}{1,2,3,4,5,6}{0.2}{6};
            \node[left=of n1,yshift=-5 mm,label=PP1,circle,fill=red] (pp1) {};
            \draw[-triangle 90 reversed] (pp1) edge[out=-0,in=180] (camkiiPSD);

            % CaMKII partial.
            \node[left=of camkiiPSD,shift=(60:-2cm)] (camkii) {};
            \CAMKII{camkiiPartial}{right=0cm of camkii}{1,2,5}{0.2}{6}

            %%%%%%%%%%%%%%%%%%%%%%%%%%%%%%%%%%%%%%%%%%%%%%%%%%%%%%%%%%%%%%%%%%%%%%%%%%%%
            % Calcium inflow  
            %%%%%%%%%%%%%%%%%%%%%%%%%%%%%%%%%%%%%%%%%%%%%%%%%%%%%%%%%%%%%%%%%%%%%%%%%%%%

            % Calcium can flow through VDCC
            \CA{ca0}{(vdcc)}{};
            \CACAM{cacam0}{(-5,0)}{0.3}
            \CAM{cam0}{-6.5,0}{0.3}

            \draw[-*,very thick] (cam0) to[midway] node (binding) {} (cacam0);
            \draw[->] (ca0) to[] (binding);

            \foreach \i in {1,2,...,5}
            {
                \coordinate (cacam) at (rnd*3-6,rnd-1)  {};
                %\CACAM{cacam\i}{cacam}{0.3}
            }


            %%%%%%%%%%%%%%%%%%%%%%%%%%%%%%%%%%%%%%%%%%%%%%%%%%%%%%%%%%%%%%%%%%%%%%
            % Channels 
            %%%%%%%%%%%%%%%%%%%%%%%%%%%%%%%%%%%%%%%%%%%%%%%%%%%%%%%%%%%%%%%%%%%%%%

            \node[below right=of cacam0] (can) {CaN};
            \node[below=of can] (i1p) {I1P/I2P};
            \node[above right=of i1p,xshift=5 mm] (ppx) {PP1/PP2};
            \node[left=of i1p,rotate=-90] (pka) {PKA};

            %%%%%%%%%%%%%%%%%%%%%%%%%%%%%%%%%%%%%%%%%%%%%%%%%%%%%%%%%%%%%%%%%%%%%%
            % Transportation of AMPA.
            %%%%%%%%%%%%%%%%%%%%%%%%%%%%%%%%%%%%%%%%%%%%%%%%%%%%%%%%%%%%%%%%%%%%%%

            \draw[->,decorate,decoration={shape backgrounds,shape=dart,shape size=2 mm}] 
            (camkiiPartial) -- (camkiiPSD);

            %\draw[-*] (ca0) edge[ ]  (cacam0);

            \draw[-*] (cacam0) edge[ ] (can);
            \draw[-triangle 90 reversed] (can) edge[ ] (i1p);
            \draw[-triangle 90 reversed] (i1p) edge[ ] (ppx);
            \draw[-triangle 90 reversed] (pka) edge[ ] (i1p);

            % Phosphorylation
            \draw[-*] (cacam0) edge[ ] node[midway] (phospho) {} (camkiiPartial);
            \draw[-latex] (camkiiPartial.north) edge[out=90,in=120] (phospho);
            \draw[-triangle 90 reversed] (ppx) edge[] (camkiiPartial);
            \draw[-triangle 90 reversed] (cacam0) edge[bend right] (pka.west);

            %% Transportation of AMPA channels.
            \AMPA{a0}{(2.8,-2)}{1}{0.1}{10};
            \CAMKII{camkiiCyt1}{above left=1cm of a0}{1,2,3,4,5,6,7}{0.2}{7}
            \draw[-triangle 90 reversed] (ppx) edge[] (camkiiCyt1);

            % CaMKII help translocating AMPA channels via Ras/MaP-K. 
            \AMPA{a1}{(-2,0.8)}{1}{0.15}{90};

            % CaMKII phosphorylated AMPA thus increasing their conductance.
            \CAMKII{camkiiCyt2}{right=0.3 cm of n0.south}{1,2,3,4,5,6,7}{0.2}{7}

            \node at ([yshift=10 mm]a1) {};
            \AMPA{a2}{(2.8,1)}{1}{0.1}{45};
            \draw[-*] (camkiiCyt2) edge[bend right] (a1);

            \draw[->,
            % ,decorate
            , decoration={footprints,foot length=10 pt,stride length=15pt}
            ] ($(a0)+(0.5,.5)$)  to[bend right] node[below,sloped,midway,yshift=-5 mm] (transport) 
            {\small MAPK} ([xshift=3mm]a2);

            \draw[-*] (camkiiCyt1) -- (transport);
        \end{scope}
        \end{tikzpicture}    
    };

    \node (synapse) at (0,0)
    {
        \begin{tikzpicture}[scale=1, rotate=0, every node/.style={} ]
            \PreSynSide{b1}

            % Post synaptic side.
            \begin{scope}[ local bounding box=c1
                    , anchor=north east
                    , at={(b1.south)}
                    , xshift=4cm, yshift=-2.8cm
                    , rotate=0
                ]
                \draw[color=red!40, very thick, fill=red!15] plot[smooth] coordinates { 
                    (-3,-8) (-3,-4) (-7,-2) (-7,1) 
                    (0,1) (2,1) (3,1) (2.5,-3.1) 
                    (-1,-3.5) (-1,-8) 
                };

                %% PSD
                %\draw[draw=black, dashed, fill=red!30] plot[smooth] coordinates { 
                %    (-6.5,0) (-6.3,1.1) (0,1.05) (2,1) (2.5,1) (2,0) (-4,0.5) (-6.5,0)
                %};
            \end{scope}

            % Add all
            \node[fill=blue!10,circle,draw=blue] (int) at (2,-6.25) {};

            % switches
            \edef\CLR{blue!80}
            \begin{scope}[rotate=0]
                \foreach[count=\i] \x in {-1.5, 2.25, 6}
                {
                    \pgfmathsetmacro{\LW}{0.5+rnd}
                    \pgfmathsetmacro{\AA}{rnd}
                    \ifdim\AA pt<0.4pt
                        \ctikzset{bipoles/pushbutton/height=1}
                        \draw[line width=\LW pt, \CLR, smooth] (\x,-2) -- (\x,-2.5) 
                            to[push button, color=blue] (\x,-4);
                    \else
                        \ctikzset{bipoles/pushbutton/height=0.5}
                        \draw[line width=\LW pt, \CLR, smooth] (\x,-2) -- (\x,-2.5) 
                            to[push button, color=blue] (\x,-4);
                    \fi

                    \draw[{Arc Barb[reversed,width=5mm,length=1mm]}-,line width=\LW pt, \CLR] 
                        (\x, -1.8) -- (\x,-2);

                    % path.
                    \pgfmathsetmacro\xnew{2.25+(\i-2)}
                    \draw[color=\CLR, thick ] plot[smooth] coordinates {
                        (\x,-4) (\xnew,-5) (int)
                    };
                }


                % integrators
                \edef\x{0.5}
                \draw[color=\CLR,] (\x,-4) node[circle, above, inner sep=2mm,
                    draw] {\large $\int$} to[bend left] (int.south);

                \edef\x{4}
                \draw[color=\CLR] (\x,-4) node[circle, above, inner sep=2mm,
                    draw] {\large $\int$} to[bend right] (int.south);

                % Redraw to hide the overlap.
                \node[fill=blue!10,circle,draw=blue] (int) at (2,-6.25) {\Huge $\sum$};
            \end{scope}

        \end{tikzpicture}	
    };

    \edef\CylWidth{1.5}
    \edef\SUSIZE{0.15}
    \edef\figW{10}
    \edef\fW{\figW}

    %\node (psd) [xshift=0cm, yshift=0.4cm]
    %{
    %    % PSD in cylinder with diffusion.
    %    \begin{tikzpicture}[baseline]
    %        % Draw camkii rings.
    %        \foreach \i/\x in {-4/1, 0/2, 4/3}
    %        {
    %            \node[ draw=red, dotted, thick, circle
    %                , minimum size=\CylWidth cm] (switch\x)  at (\i,0) { };

    %            \foreach \j in {1,...,6}
    %            {
    %                \edef\activeSE{random(1,7),random(1,7),random(1,7)
    %                ,random(1,7),random(1,7),random(1,7),random(1,7)}
    %                % Each switch has some camkii here.
    %                \coordinate (camkii0) at (\i+0.4*rand,0.3*\CylWidth*rand);
    %                \begin{scope}[rotate=30*rand]
    %                    \pgfmathsetmacro{\NumberOfSuInRing}{random(6,7)}
    %                    \CAMKIIWithoutHub{c0}{camkii0}{\activeSE}{\SUSIZE}{\NumberOfSuInRing};
    %                \end{scope}
    %            }
    %        }

    %        % Draw PSD over switches.
    %        \node[cylinder, minimum width=\CylWidth cm, minimum height=\figW cm
    %        , draw=black, dashed, fill=red, opacity=0.2, inner sep=1mm
    %        ] (psd) at (switch2) {};

    %        \foreach \j in {1,2,...,9}
    %        {
    %            \node[shade,ball color=red, fill=red, circle
    %                , minimum size=\SUSIZE cm ,inner sep=1pt] (su\j) 
    %                at (\fW/2.1*rand,0.4*\CylWidth*rand) {};
    %            \pgfmathsetmacro{\Ang}{(rand>0)?0:180}
    %            \draw[-latex,color=red,very thin] (su\j) -- ++(\Ang:3mm);
    %        }

    %        \foreach \j in {1,2,...,9}
    %        {
    %            \node[shade, ball color=blue, fill=blue, circle, minimum size=\SUSIZE cm, inner sep=1pt] 
    %                (su\j) at (\fW/2.1*rand,0.4*\CylWidth*rand) { };
    %            \pgfmathsetmacro{\Ang}{(rand>0)?0:180}
    %            \draw[-latex,color=blue,very thin] (su\j) -- ++(\Ang:3mm);
    %        }

    %    \end{tikzpicture} 
    %};

    %% Bistable trajectory.
    %\node[below=2cm of psd, xshift=0cm, yshift=0cm] (bistable) 
    %{
    %    \begin{tikzpicture}[scale=1]
    %        \begin{axis}[
    %            xlabel=Time (year), ylabel={$\Sigma$}
    %            , ylabel style={font=\Large}
    %            , width=0.75*\figW cm, height=\CylWidth cm, scale only axis
    %            , ytick={0,18}
    %            , yticklabels={\texttt{OFF},\texttt{ON}}
    %            , axis lines*=left
    %            , xmin=0, xmax=6
    %            ]
    %            \addplot [color=blue] gnuplot [ raw gnuplot ] {
    %                plot
    %                "../elifeFigure5/CaMKII-6+PP1-27+L-125e-9+N-3+diff-1e-12.dat_processed.dat"
    %                using (column("time")/3600/365):"CaMKII*" every 10;
    %            };
    %        \end{axis}
    %    \end{tikzpicture}
    %};

    % draw arrow from synapse to PSD.
    % \draw[-latex, red, very thick] ($(synapse.center)+(2.5,2.5)$) to[bend left] (psd.west);

\end{tikzpicture}	
\end{document}

