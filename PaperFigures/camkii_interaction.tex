\RequirePackage{luatex85}
\documentclass[crop,tikz]{standalone}
\usetikzlibrary{graphs,graphdrawing}
\usegdlibrary{layered,force}

\begin{document}

\begin{tikzpicture}[scale=1
    , every node/.style={}
    %, layered layout
    , spring layout
    , remember picture
    ]

    \node[ label=center:CaMKII* ]  (camkii_active) {};
    \node[ draw, inner sep=0] (camkii_bound) {};

    \node[ label=center:$Ca^{++}$ ] (ca) {};
    \node[ label=center:$CaM$ ] (cam) {};
    \node[ inner sep=0pt , draw] (cacam) {};
    \node[ label=center:Ca.CaM ] (cacamcplx) {};

    \node[ label=center:PP1 ] (pp1) {};
    \node[ inner sep=0pt , draw] (pp1_camkii_cplx) {};


%% Using graph is not a great idea for drawing the reactions.
%%    \graph[ use existing nodes
%%        , layered layout
%%        %, spring electrical layout
%%        %, level distance = 1cm 
%%        , downsize ratio=1
%%        %, node distance = 1cm
%%        %, sibling distance = 2cm
%%        , vertical=3 to 4
%%        %, grow right=2cm
%%        , edges = {thick, rounded corners}  ] 
%%    {
%%        %{ "$ca^{++}$", CaM }  --  "Ca/CaM";
%%        %{ "Ca/CaM", "CaMKII" } -- "CaMKII*";
%%        {ca, cam} -- cacam ->[] cacamcplx;
%%        {cacamcplx, camkii} -- camkii_bound -> camkii_active ;
%%
%%        {pp1, camkii_active } -- pp1_camkii_cplx -> camkii;
%%    };
    
\end{tikzpicture}    

\end{document}

