%=====================================================================================
%
%       Filename:  figure_discussion_1.tex
%
%    Description:  Figure showing how CaMKII/PP1 property changes as we move
%    towards cleft.
%
%        Version:  1.0
%        Created:  Wednesday 04 April 2018
%       Revision:  none
%
%         Author:  Dilawar Singh (), dilawars@ncbs.res.in
%   Organization:  NCBS Bangalore
%      Copyright:  Copyright (c) 2018, Dilawar Singh
%
%          Notes:  
%
%=====================================================================================

\RequirePackage{luatex85,shellesc}
\documentclass[preview,tikz,varwidth=17.8cm,multi=false]{standalone}
\usepackage{pgfplots}
\usetikzlibrary{calc,positioning}
\usepackage[sfdefault]{FiraSans}
\usepackage[small,euler-digits]{eulervm}
\usepackage{siunitx}
\usepgfplotslibrary{units,fillbetween}

\usepackage[citestyle=alphabetic]{biblatex}
\addbibresource{../bibliography.bib}


\begin{document}

\def\figW{17.8}
\pgfmathsetmacro{\colW}{0.45*\figW}
\def\figH{5}

\begin{tikzpicture}[scale=1]

    \begin{axis}[ name=figA
       , anchor=north west
       , xlabel=Distance from cleft
       , x unit=nm
       , ylabel={\#CaMKII Holoenzymes}
       , xmin = 0, xmax = 715
       , width = \colW cm, height = \figH cm
       , enlargelimits
       , title={bin size=\SI{4}{\nano\meter}, PSD radius=\SI{50}{\nano \meter}}
    ]
    
    \path[name path=y0] (axis cs:0,0) -- (axis cs:0,30);
    \path[name path=y100] (axis cs:100,0) -- (axis cs:100,30);

    \addplot[fill=black!10] fill between [of=y0 and y100 ];

    \addplot [color=blue,thick,mark=o,smooth] gnuplot [ raw gnuplot ] {
        set datafile separator ",";
        plot "../data/jneuro.petersen2003-fig7.csv" using 1:2;
    };
    \addplot [color=blue,domain=100:715,samples=154] {1};
    \node[above] at (axis cs:50,20) {\footnotesize PSD};
    \node[above] at (axis cs:200,20) {\footnotesize Cytosol};
    \end{axis}
\end{tikzpicture}


\end{document}


