%%=====================================================================================
%%
%%       Filename:  synapse_signalling.tex
%%
%%    Description:  Signalling pathway in Synapse.
%%
%%        Version:  1.0
%%        Created:  04/30/2017
%%       Revision:  none
%%
%%         Author:  Dilawar Singh (), dilawars@ncbs.res.in
%%   Organization:  NCBS Bangalore
%%      Copyright:  Copyright (c) 2017, Dilawar Singh
%%
%%          Notes:  
%%                
%%=====================================================================================

\RequirePackage{luatex85}
\documentclass[crop,tikz]{standalone}
\usepackage{pgfplots}
\usetikzlibrary{calc,graphs,graphdrawing,fit,positioning}
\usetikzlibrary{decorations,decorations.footprints,decorations.shapes}
\usetikzlibrary{shapes,arrows,arrows.meta}
\usetikzlibrary{decorations.markings}
\usegdlibrary{layered}
\usepackage{xstring}
\pgfplotsset{compat=newest}

% \CAMKIIRING{label}{x}{y}{list_of_phosphorylated_subunits}{radius_of_subunit}{no_of_subunits}


\newcommand{\tstar}[5]{% x, y, inner radius, outer radius, tips,
    \pgfmathsetmacro{\starangle}{360/#5}
    \draw[draw=none,fill=blue!40] [xshift=#1cm,yshift=#2cm](0:#3)
    \foreach \x in {1,...,#5}
    { -- (\starangle/2+\x*\starangle-\starangle/2:#4) -- (#4+\x*\starangle:#3)
    }
    -- cycle;
}

\newcommand{\TSTAR}[3]{% pos, radius, tips,
    \node[star,star points=#3,star point ratio=0.3,fill=blue!40,minimum
    size=#2cm,] (inner#3) at (#1) {};
}


\newcommand{\CAMKIIRING}[6] { % name, x, y, indices_of_red_balls, size, tips 
    % get the theta for given number of subunits.
    \pgfmathsetmacro{\theta}{360/#6};
    \pgfmathsetmacro{\r}{0.5*#5/sin(\theta/2)};
    \tstar{#2}{#3}{\r/5}{\r}{#6};
    \begin{scope}[ ]
        \def\fitlist{};
        \foreach \i in {1,...,#6}
        {
            \IfSubStr {#4} {\i}
            {
                \node[ball color=red,circle,shading=ball,minimum size=#5cm] (r\i) at
                ($(\i*\theta:\r cm)+(#2,#3)$) {};
            }
            {
                \node[draw=none,ball color=blue,circle,minimum size=#5cm](r\i) at
                ($(\i*\theta:\r cm)+(#2,#3)$) {}; 
            }
            \xdef\fitlist{\fitlist(r\i)};
        };
        % inner subunit.

        \node[circle,fit=\fitlist,] (#1) {};
        %\node[] at (#2,#3) {#1};
    \end{scope}
}

\newcommand\SUBUNIT[4]{ % name, pos, radius, color
    \node[draw=none,ball color=#4,inner sep=0,circle,minimum size=#3 cm] (#1) at #2 {}; 
}

\newcommand{\CAMKII}[5] { % name, position, indices_of_red_balls, size, tips 
    % get the theta for given number of subunits.
    \node (#1) at (#2) {};
    \begin{scope}[ ]
        \pgfmathsetmacro{\theta}{360/#5};
        \pgfmathsetmacro{\r}{0.5*#4/sin(\theta/2)};
        \def\fitlist{};

        \node[draw=none] (#1_root) at (#1) {};
        % inner subunit.
        \TSTAR{#1_root.center}{0.4*\r}{#5};

        \foreach \i in {1,...,#5}
        {
            \IfSubStr {#3} {\i}
            {
                \node[draw=none,inner sep=0,ball color=red,circle,minimum size=#4cm] (r\i) at
                ($(\i*\theta:\r cm)+(#1_root)$) {};
            }
            {
                \node[draw=none,ball color=blue,inner sep=0,circle,minimum size=#4cm](r\i) at
                ($(\i*\theta:\r cm)+(#1_root)$) {}; 
            }
            \xdef\fitlist{\fitlist(r\i)};
        };
        \node[fit=\fitlist,circle,inner sep=0] (#1) {};
    \end{scope}
}

\newcommand{\CAMKIIWithoutHub}[5] { % name, position, indices_of_red_balls, size, tips 
    % get the theta for given number of subunits.
    \node (#1) at (#2) {};
    \begin{scope}[ ]
        \pgfmathsetmacro{\theta}{360/#5};
        \pgfmathsetmacro{\r}{0.5*#4/sin(\theta/2)};
        \def\fitlist{};

        \node[draw=none] (#1_root) at (#1) {};

        \foreach \i in {1,...,#5}
        {
            \IfSubStr {#3} {\i}
            {
                \node[draw=none,inner sep=0,ball color=red,circle,minimum size=#4cm] (r\i) at
                ($(\i*\theta:\r cm)+(#1_root)$) {};
            }
            {
                \node[draw=none,ball color=blue,inner sep=0,circle,minimum size=#4cm](r\i) at
                ($(\i*\theta:\r cm)+(#1_root)$) {}; 
            }
            \xdef\fitlist{\fitlist(r\i)};
        };
        \node[fit=\fitlist,circle,inner sep=0] (#1) {};
    \end{scope}
}


\newcommand{\CACAM}[3] { % name, (x, y), size 
    % A node is created with given name which fit all others.
    \node (#1) at (#2) { };
    \edef\name{#1_center}
    \begin{scope}[]
        \pgfmathsetmacro{\car}{#3/2}
        \node[star,star points=4,fill=red,minimum size=#3 cm,inner sep=0pt] 
            (\name) at (#1) {};

        \foreach \x in {1,...,4} 
        {
            \pgfmathsetmacro{\theta}{360/4*\x};
            \node[inner sep=0pt,ball color=yellow,circle,minimum size=\car cm] 
            (ca\x) at ($(#2)+(\theta:\car cm)$) {};
        }
        \node[circle,inner sep=0,fit=(ca1) (ca2) (ca3) (ca4) (\name)] (#1)  {};
    \end{scope}
}

\newcommand{\CAM}[3] { % name, (x, y), size 
    \node (#1) at (#2) { };
    \begin{scope}[]
        \pgfmathsetmacro{\car}{#3/2}
        \node[star,star points=4,star point ratio=2
            , fill=red,minimum size=#3 cm,inner sep=0pt] 
            (rcam) at (#1) {};
    \end{scope}
}

% NMDA receptors and other.
% \NMDAOLD}{(x,y)}{width}{gap}{rotation}
\newcommand{\NMDAOLD}[5] {
    \pgfmathsetmacro\rwidth{#3/5.0}
    \pgfmathsetmacro\gap{0.1+#4}

    \edef\ang{#5}
    \node[minimum height=#3 cm,minimum width=\gap cm] (#1) at #2 {};
    \begin{scope}[rotate around={(\ang:#2)}]
        \foreach \xshift/\name in {\gap/#1_right,-\gap/#1_left}
        {
            \node[draw=blue,fill=red!20,rectangle, inner sep=0pt
                , decorate, decoration={random steps,amplitude=1pt,segment length=1pt}
                , minimum height=#3cm ,minimum width=\rwidth cm
            ] (\name) at ([xshift=\xshift cm]#1.east) {};
        }
    \end{scope}
}

% \AMPA{name}{(x0,y0)}{height}{gap or opening size}{rotation}
\newcommand{\AMPA}[5] {
    \pgfmathsetmacro{\rwidth}{#3/5.0}
    \pgfmathsetmacro\rwidth{#3/5.0}
    \pgfmathsetmacro\gap{0.1+#4}
    \edef\ang{#5}

    \node (#1) at #2 {};
    \begin{scope}[rotate around={(\ang:#2)}]
        \foreach \i in {\gap,0}
        {
            \node[minimum height=#3 cm,minimum width=\gap cm
                ,transform shape   % necessary when within scope
                ,cylinder,fill=red
            ] (#1_\i) at ([yshift=\i cm]#1) {};
        }
    \end{scope}
}

% \CA{label}{coordinate}{label}
\newcommand{\CA}[3] {
    \node (#1) at #2 {};
    \node[shading=ball,circle,ball color=yellow,inner sep=0,minimum size=2 mm]
        at (#1) { };
}


%%%%%%%%%%%%%%%%%%%%%%%%%%%%%%
%% Chemical equilibrium arrow 

\newdimen\arrowsize
\newdimen\mylw
\pgfkeys{/my arrows/chemeq/.style={draw,thick,double distance=3pt,onearc-onearc}}
\pgfkeys{/my arrows/size/.code={\pgfsetarrowoptions{onearc}{#1}}}
\def\myalw{1pt}
\pgfarrowsdeclare{onearc}{onearc}{%
  \mylw=\myalw
  \pgfarrowsleftextend{-\pgfgetarrowoptions{onearc}-.5\mylw}
  \pgfarrowsrightextend{1pt}
}{%
  \pgfsetdash{}{0pt}
  \mylw=\pgflinewidth
  \pgfsetlinewidth{\myalw}
  \advance\arrowsize by.5\pgflinewidth
  \pgfpathmoveto{\pgfpoint{-\pgfgetarrowoptions{onearc}}{-\pgfgetarrowoptions{onearc}-.5\mylw}}%
  \pgfpatharc{180}{90}{\pgfgetarrowoptions{onearc}}
  \pgfusepathqstroke
}


%  \PHOSPHO{name}{location}
\newcommand\PHOSPHO[2]
{
    \node[circle,fill=yellow,inner sep=0pt] (#1) at (#2) {\tiny P};
}


\newcommand{\CAMKIIHOLOENZYME}[6] { % name, x, y, indices_of_red_balls, size, tips 
    % get the theta for given number of subunits.
    \begin{scope}[ ]
        \pgfmathsetmacro{\theta}{360/#6};
        \pgfmathsetmacro{\r}{0.5*#5/sin(\theta/2)};
        \def\fitlist{};
        \foreach \i in {1,...,#6}
        {
            \IfSubStr {#4} {\i}
            {
                \node[ball color=red,circle,shading=ball,minimum size=#5cm] (r\i) at
                ($(\i*\theta:\r cm)+(#2,#3,0)$) {};
            }
            {
                \node[draw=none,ball color=blue,circle,minimum size=#5cm](r\i) at
                ($(\i*\theta:\r cm)+(#2,#3,0)$) {}; 
            }
            \xdef\fitlist{\fitlist(r\i)};
        };

        % inner hub.
        % \tstar{#2}{#3}{\r/5}{\r}{#6};

        \foreach \i in {1,...,#6}
        {
            \IfSubStr {#4} {\i}
            {
                \node[ball color=red,circle,shading=ball,minimum size=#5cm] (r\i) at
                ($(\i*\theta:\r cm)+(#2,#3,\r)$) {};
            }
            {
                \node[draw=none,ball color=blue,circle,minimum size=#5cm](r\i) at
                ($(\i*\theta:\r cm)+(#2,#3,\r)$) {}; 
            }
            \xdef\fitlist{\fitlist(r\i)};
        };
        \node[circle,fit=\fitlist,] (#1) {};
        %\node[] at (#2,#3) {#1};
    \end{scope}
}


%%=====================================================================================
%%
%%       Filename:  channel.tex
%%
%%    Description:  Here is a channel.
%%
%%        Version:  1.0
%%        Created:  05/04/2017
%%       Revision:  none
%%
%%         Author:  Dilawar Singh (), dilawars@ncbs.res.in
%%   Organization:  NCBS Bangalore
%%      Copyright:  Copyright (c) 2017, Dilawar Singh
%%
%%          Notes:  
%%                
%%=====================================================================================

% CHANNEL{node}{height}{width}{slope}{color}
\newcommand\CHANNEL[5]{%
    \pgfmathsetmacro\a{#2*0.5}
    \pgfmathsetmacro\b{\a*0.3}

    \pgfmathsetmacro\w{#2}
    \pgfmathsetmacro\h{#3}

    \pgfmathsetmacro\al{\a-(0.5*\h*tan(#4))}
    \pgfmathsetmacro\bl{\al*0.3}

    % Plot two funnels.
    \begin{scope}[ shift={(#1)}, ]

        % center of lower and upper arc
        \coordinate (#1_c1) at (0,0);
        \coordinate (#1_c2) at (0,-\h);

        % Upper elliptic arc
        \pgfmathsetmacro\xarc{\a*cos(-45)}
        \pgfmathsetmacro\yarc{\b*sin(-45)}

        % Lower elliptic arc
        \pgfmathsetmacro\xlarc{\al*cos(-45)}
        \pgfmathsetmacro\ylarc{\bl*sin(-45)-\h}


        \path[opacity=1,fill=#5] (#1_c1) -- (\xarc,\yarc) arc (-45:225:\a cm and
            \b cm) node[ ] (#1_upper_end) {};
        \path[opacity=1,fill=#5] (#1_c2) -- (\xlarc,\ylarc) arc (-45:225:\al cm and \bl cm)
            node [ ] (#1_lower_end) {};

        \path[fill=#5!30] (#1_lower_end.center) -- (#1_upper_end.center) 
            -- (#1_c1.center) -- (\xarc, \yarc) -- (\xlarc, \ylarc)
            -- (#1_c2.center) -- cycle;

        % Fill outer surface. One should also compute the arc here to make
        % surface look curvy here. Or we can just make it opaque!
        \path[fill=#5,opacity=0.5] (-\a,0) -- (-\al,-\h) --
            (#1_lower_end.center) -- (#1_upper_end.center);
        \path[fill=#5,opacity=0.5] (\a,0) -- (\al,-\h) -- (\xlarc,\ylarc) -- (\xarc,\yarc);

        % Put a hollow cylinder inside the funnel.
        \node[rotate=90,minimum height=1.05*\h cm,fill=white!50 , minimum width=2*\bl cm
            , cylinder, aspect=0.25,draw=#5!10] at ($(#1_c1.center)!0.58!(#1_c2.center)$) { };

    \end{scope}
}

%\begin{document}
%\RequirePackage{luatex85}
%\documentclass[crop,tikz]{standalone}
%\usepackage{tikz,pgfplots}
%\usetikzlibrary{shapes,calc,positioning}

%\begin{tikzpicture}[scale=1, every node/.style={} ]
    %\node (a) at (0,0) { };
    %\CHANNEL{a}{1}{0.8}{60}{green};
    %\CHANNEL{a}{0.7}{0.5}{10}{blue};

%\end{tikzpicture}    

%\end{document}


\input{actin.tex}

\newcommand\NMDA[1]{\CHANNEL{#1}{1}{0.8}{40}{red}}

\begin{document}

%\pgfmathsetseed{10}
\tikzset{
    -PO/.style = {
        decoration = { markings, 
            mark=at position -3pt with { 
                \node[draw,circle,fill=yellow,inner sep=1pt] {\tiny P}; }
        }, postaction = decorate, shorten >=3pt,
    }
}

\begin{tikzpicture}[scale=1, every node/.style={},% node distance=5mm 
    ]

    % Grid 
    %\draw[thin,step=1,gray!10] (-10,-10) grid (5,5);

    % Draw synapse.
    \edef\pdenda{(-8,-8)}
    \edef\pdendb{(-5,-8.2)}
    \edef\pnecka{(-3,-8)}
    \edef\pneckb{(-3,-4)}

    % Draw some actin.
    % First all the actin in dendrites.
    \foreach \i in {1,2,...,3}
    {
        \coordinate (start) at ([xshift=rnd,yshift=rnd](-8,-9));
        \coordinate (direcP) at ([xshift=rnd,yshift=rnd](0,-9));
        \pgfmathsetmacro\count{50+50*rnd}
        \ACTIN{a\i}{(start)}{(direcP)}{0.5pt}{\count};
    }


    \draw[color=blue,very thick] plot[smooth] coordinates { 
            \pdenda \pdendb \pnecka \pneckb (-7, -2) (-7,1) 
            (0, 1) (3,1) (2.5,-3) 
            (-1,-4) (-1,-8) (3,-8)
        };
    \node[ ] at (-5,-10) {Dendrite};


    \draw[blue,very thick] plot[smooth] coordinates { 
            (-8, -13) (-5, -13.5) (0,-13.1) (3, -13)
        };

    % draw PSD.
    \path[fill=blue!50,decorate,decoration={random steps}]
        (-6,1.2) -- (-5,1.2) -- (-4,1.2) -- (0,1.1) -- (3,1.2) 
        --  (1,0) -- (0,0) -- (-3,0) -- (-5,0) -- cycle;

    %%%%%%%%%%%%%%%%%%%%%%%%%%%%%%%%%%%%%%%%%%%%%%%%%%%%%%%%%%%%%%%%%%%%%%%%%%
    % A lot of calcium outside  and calmodulin inside.
    %%%%%%%%%%%%%%%%%%%%%%%%%%%%%%%%%%%%%%%%%%%%%%%%%%%%%%%%%%%%%%%%%%%%%%%%%%
    \foreach \i/\x/\y in {1/-7/2,2/-6.5/2.1,3/-6.2/1.9,4/-6.8/2.2}
    {
        \CA{ca\i}{(\x,\y)}{};
    }


    %%%%%%%%%%%%%%%%%%%%%%%%%%%%%%%%%%%%%%%%%%%%%%%%%%%%%%%%%%%%%%%%%%%%%%%%%%
    % Draw channels.
    % VDCC or voltage dependant calcium channel opens when membrane potential at
    % spine goes above threshold voltage of channel.
    %%%%%%%%%%%%%%%%%%%%%%%%%%%%%%%%%%%%%%%%%%%%%%%%%%%%%%%%%%%%%%%%%%%%%%%%%%
    \coordinate (vdcc) at (-6,1);
    \node[draw=blue,rotate=110,cylinder,fill=white,label=VDCC,minimum height=7 mm] 
        at (vdcc) {};

    \node (n0) at (-5,1.2) { };
    \NMDA{n0}

    \node[above=of n0] {NMDA};
    \CA{canmda}{([yshift=-3mm]n0_c1)}{};
    \CA{canmda}{([yshift=-6mm]n0_c1)}{};

    % Another NMDA.
    \node (n1) at (1,1.2) { };
    \NMDA{n1};


    %%%%%%%%%%%%%%%%%%%%%%%%%%%%%%%%%%%%%%%%%%%%%%%%%%%%%%%%%%%%%%%%%%%%%%%%%%%%
    % Calcium inflow  
    %%%%%%%%%%%%%%%%%%%%%%%%%%%%%%%%%%%%%%%%%%%%%%%%%%%%%%%%%%%%%%%%%%%%%%%%%%%%

    % Calcium can flow through VDCC
    \CA{ca0}{(vdcc)}{};
    \CA{ca1}{([yshift=-0.4cm]vdcc)}{};
    \CA{ca2}{([yshift=-1cm]vdcc)}{};

    \CACAM{cacam0}{-5.5,-0.5}{0.3}
    \CAM{cam0}{-6.5,0}{0.3}

    \draw[->,very thick] (cam0) to[midway] node (binding) {} (cacam0);
    \foreach \i in {1,2,...,5}
    {
        \coordinate (cacam) at (rnd*3-6,rnd-1)  {};
    }

    %%%%%%%%%%%%%%%%%%%%%%%%%%%%%%%%%%%%%%%%%%%%%%%%%%%%%%%%%%%%%%%%%%%%%%%%%%%%
    % CaMKIIs.
    %%%%%%%%%%%%%%%%%%%%%%%%%%%%%%%%%%%%%%%%%%%%%%%%%%%%%%%%%%%%%%%%%%%%%%%%%%%%
    % Attach a CaMKII below it.
    \CAMKII{camkiiPSD}{[xshift=-5mm]n1_c2.west}{1,2,3,4,5,6}{0.2}{6};
    
    \node[left=of n1,yshift=-5 mm,label=PP1,circle,fill=red] (pp1) {};
    \draw[-triangle 90 reversed] (pp1) edge[] (camkiiPSD);

    % CaMKII partial.
    \node[right=of cacam0,shift=(-30:5mm)] (camkii) {};
    \CAMKII{camkiiPartial}{camkii}{1,2,5}{0.2}{6}

    % CaMKII in spine neck.
    \node[shift=(-30:10mm)] (camkiineck) at \pneckb {};
    \CAMKII{camkiineck}{camkiineck}{}{0.2}{6}

    % A lot of camkii in dendritic shaft.
    \node[shift=(-90:10mm)] (camkiidend1) at \pdenda {};
    \foreach \i in {1,2,3,4,5}
    {
        \pgfmathsetmacro\theta{-20*rnd}
        \pgfmathsetmacro\r{rnd*10}
        \CAMKII{camkiineck}{[shift=(\theta:\r cm)]camkiidend1}{}{0.2}{6}
    }



    %%%%%%%%%%%%%%%%%%%%%%%%%%%%%%%%%%%%%%%%%%%%%%%%%%%%%%%%%%%%%%%%%%%%%%
    % Channels 
    %%%%%%%%%%%%%%%%%%%%%%%%%%%%%%%%%%%%%%%%%%%%%%%%%%%%%%%%%%%%%%%%%%%%%%

    \node[below right=of cacam0] (can) {CaN};
    \node[below=of can] (i1p) {I1P/I2P};
    \node[above right=of i1p,xshift=5 mm] (ppx) {PP1/PP2};
    \node[left=of i1p,rotate=-90] (pka) {PKA};

     %%%%%%%%%%%%%%%%%%%%%%%%%%%%%%%%%%%%%%%%%%%%%%%%%%%%%%%%%%%%%%%%%%%%%%
     % Transportation of AMPA.
     %%%%%%%%%%%%%%%%%%%%%%%%%%%%%%%%%%%%%%%%%%%%%%%%%%%%%%%%%%%%%%%%%%%%%%

    \draw[->,decorate,decoration={shape backgrounds,shape=dart,shape size=2 mm}] 
        (camkiiPartial) -- (camkiiPSD);

    \draw[-PO] (cacam0) to[ ] (can);
    \draw[-triangle 90 reversed] (can) edge[ ] (i1p);
    \draw[-triangle 90 reversed] (i1p) edge[ ] (ppx);
    \draw[-triangle 90 reversed] (pka) edge[ ] (i1p);
    
    % Phosphorylation
    \draw[-PO] (cacam0) to[ ] node[midway] (phospho) {} (camkiiPartial);
    \draw[-latex] (camkiiPartial.north) edge[out=90,in=120] (phospho);
    \draw[-triangle 90 reversed] (ppx) edge[] (camkiiPartial);
    \draw[-triangle 90 reversed] (cacam0) edge[bend right] (pka.west);

    %% Transportation of AMPA channels.
    \AMPA{a0}{(2.8,-2)}{1}{0.1}{10};
    \CAMKII{camkiiCyt1}{[shift=(135:2 cm)]a0}{1,2,3,4,5,6,7}{0.2}{7}
    \draw[-triangle 90 reversed] (ppx) edge[] (camkiiCyt1);

    % CaMKII help translocating AMPA channels via Ras/MaP-K. 
    \AMPA{a1}{(-2,0.8)}{1}{0.15}{90};

    % CaMKII phosphorylated AMPA thus increasing their conductance.
    \CAMKII{camkiiCyt2}{[shift=(0:5 mm)]n0_c2}{1,2,3,4,5,6,7}{0.2}{7}

    \node at ([yshift=10 mm]a1) {AMPA};
    \AMPA{a2}{(2.8,1)}{1}{0.1}{45};

    % Phosphorylates AMPA.
    \draw[-PO] (camkiiCyt2) to[bend right,right] ([shift=(-135:5mm)]a1);
        
    \draw[->,very thick
            ,decorate, decoration={footprints,foot length=10 pt,stride length=15pt}
            ] ($(a0)+(0.5,.5)$)  to[bend right] node[below,sloped,midway,yshift=-5 mm] (transport) 
            {\small MAPK} ([xshift=3mm]a2);

    \draw[-PO] (camkiiCyt1) -- (transport);

\end{tikzpicture}    

%\begin{tabular}{c c}
%    Molecule & size \\
%    CaMKII & 120 A \\
%    NMDA & 120 A  \\
%    ACTIN diameter & 50 A
%\end{tabular}

\end{document}
