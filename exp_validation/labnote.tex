\documentclass[]{article}
\usepackage[]{libertine}
\usepackage{amssymb,amsmath}
\usepackage{ifxetex,ifluatex}
\usepackage{fixltx2e} % provides \textsubscript
\ifnum 0\ifxetex 1\fi\ifluatex 1\fi=0 % if pdftex
  \usepackage[T1]{fontenc}
  \usepackage[utf8]{inputenc}
\else % if luatex or xelatex
  \ifxetex
    \usepackage{mathspec}
  \else
    \usepackage{fontspec}
  \fi
  \defaultfontfeatures{Ligatures=TeX,Scale=MatchLowercase}
\fi
% use upquote if available, for straight quotes in verbatim environments
\IfFileExists{upquote.sty}{\usepackage{upquote}}{}
% use microtype if available
\IfFileExists{microtype.sty}{%
\usepackage[]{microtype}
\UseMicrotypeSet[protrusion]{basicmath} % disable protrusion for tt fonts
}{}
\PassOptionsToPackage{hyphens}{url} % url is loaded by hyperref
\usepackage[unicode=true]{hyperref}
\hypersetup{
            pdftitle={Effect of Subunit Exchange on the Bistability of Small CaMKII/PP1 Switch},
            pdfauthor={Dilawar Singh},
            pdfborder={0 0 0},
            breaklinks=true}
\urlstyle{same}  % don't use monospace font for urls
\usepackage[right=5cm, marginparwidth=4cm]{geometry}
\usepackage{color}
\usepackage{fancyvrb}
\newcommand{\VerbBar}{|}
\newcommand{\VERB}{\Verb[commandchars=\\\{\}]}
\DefineVerbatimEnvironment{Highlighting}{Verbatim}{commandchars=\\\{\}}
% Add ',fontsize=\small' for more characters per line
\newenvironment{Shaded}{}{}
\newcommand{\KeywordTok}[1]{\textcolor[rgb]{0.00,0.44,0.13}{\textbf{#1}}}
\newcommand{\DataTypeTok}[1]{\textcolor[rgb]{0.56,0.13,0.00}{#1}}
\newcommand{\DecValTok}[1]{\textcolor[rgb]{0.25,0.63,0.44}{#1}}
\newcommand{\BaseNTok}[1]{\textcolor[rgb]{0.25,0.63,0.44}{#1}}
\newcommand{\FloatTok}[1]{\textcolor[rgb]{0.25,0.63,0.44}{#1}}
\newcommand{\ConstantTok}[1]{\textcolor[rgb]{0.53,0.00,0.00}{#1}}
\newcommand{\CharTok}[1]{\textcolor[rgb]{0.25,0.44,0.63}{#1}}
\newcommand{\SpecialCharTok}[1]{\textcolor[rgb]{0.25,0.44,0.63}{#1}}
\newcommand{\StringTok}[1]{\textcolor[rgb]{0.25,0.44,0.63}{#1}}
\newcommand{\VerbatimStringTok}[1]{\textcolor[rgb]{0.25,0.44,0.63}{#1}}
\newcommand{\SpecialStringTok}[1]{\textcolor[rgb]{0.73,0.40,0.53}{#1}}
\newcommand{\ImportTok}[1]{#1}
\newcommand{\CommentTok}[1]{\textcolor[rgb]{0.38,0.63,0.69}{\textit{#1}}}
\newcommand{\DocumentationTok}[1]{\textcolor[rgb]{0.73,0.13,0.13}{\textit{#1}}}
\newcommand{\AnnotationTok}[1]{\textcolor[rgb]{0.38,0.63,0.69}{\textbf{\textit{#1}}}}
\newcommand{\CommentVarTok}[1]{\textcolor[rgb]{0.38,0.63,0.69}{\textbf{\textit{#1}}}}
\newcommand{\OtherTok}[1]{\textcolor[rgb]{0.00,0.44,0.13}{#1}}
\newcommand{\FunctionTok}[1]{\textcolor[rgb]{0.02,0.16,0.49}{#1}}
\newcommand{\VariableTok}[1]{\textcolor[rgb]{0.10,0.09,0.49}{#1}}
\newcommand{\ControlFlowTok}[1]{\textcolor[rgb]{0.00,0.44,0.13}{\textbf{#1}}}
\newcommand{\OperatorTok}[1]{\textcolor[rgb]{0.40,0.40,0.40}{#1}}
\newcommand{\BuiltInTok}[1]{#1}
\newcommand{\ExtensionTok}[1]{#1}
\newcommand{\PreprocessorTok}[1]{\textcolor[rgb]{0.74,0.48,0.00}{#1}}
\newcommand{\AttributeTok}[1]{\textcolor[rgb]{0.49,0.56,0.16}{#1}}
\newcommand{\RegionMarkerTok}[1]{#1}
\newcommand{\InformationTok}[1]{\textcolor[rgb]{0.38,0.63,0.69}{\textbf{\textit{#1}}}}
\newcommand{\WarningTok}[1]{\textcolor[rgb]{0.38,0.63,0.69}{\textbf{\textit{#1}}}}
\newcommand{\AlertTok}[1]{\textcolor[rgb]{1.00,0.00,0.00}{\textbf{#1}}}
\newcommand{\ErrorTok}[1]{\textcolor[rgb]{1.00,0.00,0.00}{\textbf{#1}}}
\newcommand{\NormalTok}[1]{#1}
\usepackage{longtable,booktabs}
% Fix footnotes in tables (requires footnote package)
\IfFileExists{footnote.sty}{\usepackage{footnote}\makesavenoteenv{long table}}{}
\usepackage{graphicx,grffile}
\makeatletter
\def\maxwidth{\ifdim\Gin@nat@width>\linewidth\linewidth\else\Gin@nat@width\fi}
\def\maxheight{\ifdim\Gin@nat@height>\textheight\textheight\else\Gin@nat@height\fi}
\makeatother
% Scale images if necessary, so that they will not overflow the page
% margins by default, and it is still possible to overwrite the defaults
% using explicit options in \includegraphics[width, height, ...]{}
\setkeys{Gin}{width=\maxwidth,height=\maxheight,keepaspectratio}
\IfFileExists{parskip.sty}{%
\usepackage{parskip}
}{% else
\setlength{\parindent}{0pt}
\setlength{\parskip}{6pt plus 2pt minus 1pt}
}
\setlength{\emergencystretch}{3em}  % prevent overfull lines
\providecommand{\tightlist}{%
  \setlength{\itemsep}{0pt}\setlength{\parskip}{0pt}}
\setcounter{secnumdepth}{5}
% Redefines (sub)paragraphs to behave more like sections
\ifx\paragraph\undefined\else
\let\oldparagraph\paragraph
\renewcommand{\paragraph}[1]{\oldparagraph{#1}\mbox{}}
\fi
\ifx\subparagraph\undefined\else
\let\oldsubparagraph\subparagraph
\renewcommand{\subparagraph}[1]{\oldsubparagraph{#1}\mbox{}}
\fi

% set default figure placement to htbp
\makeatletter
\def\fps@figure{htbp}
\makeatother

\usepackage{pgfplots,siunitx}
\DefineVerbatimEnvironment{Highlighting}{Verbatim}{commandchars=\\\{\},fontsize=\scriptsize}
\usepackage{subfig}
\AtBeginDocument{%
\renewcommand*\figurename{Figure}
\renewcommand*\tablename{Table}
}
\AtBeginDocument{%
\renewcommand*\listfigurename{List of Figures}
\renewcommand*\listtablename{List of Tables}
}
\usepackage{float}
\floatstyle{ruled}
\makeatletter
\@ifundefined{c@chapter}{\newfloat{codelisting}{h}{lop}}{\newfloat{codelisting}{h}{lop}[chapter]}
\makeatother
\floatname{codelisting}{Listing}
\newcommand*\listoflistings{\listof{codelisting}{List of Listings}}

\title{Effect of Subunit Exchange on the Bistability of Small CaMKII/PP1 Switch}
\author{Dilawar Singh}
\providecommand{\institute}[1]{}
\institute{NCBS Bangalore}
\date{\today}

\begin{document}
\maketitle

\section{Results}\label{results}

\subsection{Subunit exchange does not change the residence time of
active state}\label{sec:res1}

We do not see any significant effect of subunit exchange on residence
time. The blue and red curve representing switch with subunit-exchange
and without subunit exchange respectively are more or less same.

Also see secs.~\ref{sec:va}, \ref{sec:vb}, \ref{sec:vc}.

---------------------------------------------\textbar{}------------------------------------
\includegraphics[width=8.00000cm]{./figure_camkii12_voxel1.pdf}
\textbar{}
\includegraphics[width=8.00000cm]{./figure_camkii12_voxel2.pdf}
CaMKII=12, Voxel=1 \textbar{} CaMKII=12, Voxels=2

\subsection{As the number of voxels increases, the switching rate
increases.}\label{as-the-number-of-voxels-increases-the-switching-rate-increases.}

Though the residence does not change with subunit exchange, we see an
increases in number of switching. See 2 subplot of figure in
sec.~\ref{sec:res1}. The number of transitions increases from 14 to 40.
\textbf{This is not a property of subunit exchange but of number of
voxels.} In other words, this is an indication of fact that multiple
bistables kept together are likely to switch faster.

\begin{figure}
\centering
\includegraphics{./figure_transition_rate_vs_voxels.pdf}
\caption{}
\end{figure}

\newpage 

\hypertarget{sec:experiment}{\section{Experiment}\label{sec:experiment}}

We compute following two parameters to measure the bistability of
switch.

\begin{itemize}
\tightlist
\item
  Number of \texttt{OFF} to \texttt{ON} (or \texttt{ON} to \texttt{OFF})
  transitions in time-series.
\item
  Time spent in \texttt{ON} state.
\end{itemize}

For various values of CaMKII rings and PP1 molecules in system, we run
the simulations for two situations\textgreater{}

\begin{enumerate}
\def\labelenumi{\arabic{enumi}.}
\tightlist
\item
  When no subunit exchange is enabled.
\item
  When subunit exchange is enabled.
\end{enumerate}

\subsection{Simulation structure}\label{simulation-structure}

Following bash script fragment shows the structure of this simulation.
An equivalent of this is implemented in CMake.
\marginpar{\scriptsize Some parameters are not shown}

\begin{Shaded}
\begin{Highlighting}[]
\VariableTok{DIFF_CONST=}\NormalTok{1e-13}
\KeywordTok{for} \ExtensionTok{ENABLE_DIFF}\NormalTok{ in ON OFF}\KeywordTok{;} \KeywordTok{do}
    \KeywordTok{for} \ExtensionTok{CAMKII}\NormalTok{ in 8 12}\KeywordTok{;} \KeywordTok{do}
        \KeywordTok{for} \ExtensionTok{pp1}\NormalTok{ in }\VariableTok{$(}\FunctionTok{seq}\NormalTok{ 30 2 50}\VariableTok{)}\KeywordTok{;} \KeywordTok{do}
            \KeywordTok{for} \ExtensionTok{voxel}\NormalTok{ in 1 2 3}\KeywordTok{;} \KeywordTok{do}
                \ExtensionTok{python}\NormalTok{ camkii_pp1_scheme.py --camkii }\VariableTok{$\{CAMKII\}}\NormalTok{ --pp1 }\VariableTok{$\{pp1\}}\NormalTok{ \textbackslash{}}
\NormalTok{                    --nvoxels }\VariableTok{$\{voxel\}}\NormalTok{ \textbackslash{}}
\NormalTok{                    --diff-dict }\StringTok{"\{ 'x' : }\VariableTok{$\{DIFF_CONST\}}\StringTok{, 'y' : }\VariableTok{$\{DIFF_CONST\}}\StringTok{ \}"}
                    \ExtensionTok{--enabled-diffusion}\NormalTok{=}\VariableTok{$\{ENABLE_DIFF\}}
            \KeywordTok{done}
        \KeywordTok{done}
    \KeywordTok{done}
\KeywordTok{done}
\end{Highlighting}
\end{Shaded}

\subsection{Data Analysis}\label{data-analysis}

Raw data from each simulation is analyzed to get parameters discussed in
section \protect\hyperlink{sec:experiment}{Experiment}.

\subsection{With 1 voxel}\label{sec:va}

When only 1 voxel is used, diffusion has no effect, no matter what value
of diffusion coefficient is used. This is because diffusion is
implemented as cross-voxel reactions. The subunits are released and
picked up but they do not diffusion between voxels.

We see almost same behaviour in both cases \emph{though number of
transitions seems to be increasing at larger value of PP1}. At large
value of PP1, the switch is hardly spending any time in \texttt{ON}
state.

\marginpar{\scriptsize At PP1=40, number of transitions are higher when subunit
exchange is enabled.}

\begin{longtable}[]{@{}ll@{}}
\toprule
\includegraphics[width=8.00000cm]{./figure_camkii8_voxel1.pdf} &
\includegraphics[width=8.00000cm]{./figure_camkii12_voxel1.pdf}\tabularnewline
CaMKII=8, Voxels=1 & CaMKII=12, Voxels=1\tabularnewline
\bottomrule
\end{longtable}

In this case, two voxels were created and diffusion was setup up with
diffusion coefficient of \SI{1e-13}{\meter ^2 \per \second}

\subsection{With 2 voxels}\label{sec:vb}

\begin{longtable}[]{@{}ll@{}}
\toprule
\includegraphics[width=8.00000cm]{./figure_camkii8_voxel2.pdf} &
\includegraphics[width=8.00000cm]{./figure_camkii12_voxel2.pdf}\tabularnewline
CaMKII=8, Voxels=2 & CaMKII=12, Voxels=2\tabularnewline
\bottomrule
\end{longtable}

\subsection{With 3 voxels}\label{sec:vc}

\begin{longtable}[]{@{}ll@{}}
\toprule
\includegraphics[width=8.00000cm]{./figure_camkii8_voxel3.pdf} &
\includegraphics[width=8.00000cm]{./figure_camkii12_voxel3.pdf}\tabularnewline
CaMKII=8, Voxels=3 & CaMKII=12, Voxels=3\tabularnewline
\bottomrule
\end{longtable}

\newpage

\section{Appendix}\label{appendix}

\begin{figure}
\centering
\includegraphics{./figure_sample_trajectories.pdf}
\caption{Sample trajectories. One from each case. When number of voxels
are increased, switching activity increases.}\label{fig:sample_trajs}
\end{figure}

\end{document}
